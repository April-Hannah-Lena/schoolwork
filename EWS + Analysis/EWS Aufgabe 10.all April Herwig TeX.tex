\documentclass[draft]{article}
\usepackage[utf8]{inputenc}
\usepackage{amsmath}
\usepackage{bbm}
\setlength{\parindent}{0pt}

\title{EWS Blatt 10}
\author{April Herwig}

\begin{document}

\maketitle

a) Die Funktion $\mathbbm{1}_{(-\infty, x]} (X_i)$ ist äquivalent zu einer Bernoulli verteilten Zufalsvariable $Y_i \sim Bernoulli(F(x)) \in \mathcal{L}^2$. Damit ist für die Summe der unabhängig, identisch verteilten Stichproben: 
\begin{align*}
    &P(|\sum\limits_{i=1}^{n}\mathbbm{1}_{(-\infty, x]} (X_i)-F(x)|>\epsilon) = P(|\sum\limits_{i=1}^{n} Y_i-F(x)|>\epsilon) \rightarrow 0,
\end{align*}
stochastisch nach dem schwachem Gesetz der Großen Zahlen.
\newline
\newline
\newline
b) Es ist: 
\begin{align*}
    F_\theta^{-1}(p) \leq x \Leftrightarrow & p < F_\theta(x) \\
    & p = F_n(x) \Leftrightarrow F_n(x) \leq F_\theta(x) .
\end{align*}
Betrachte also $F_n^{-1}(p) - F_\theta^{-1}(p) > \epsilon$
\begin{align*}
    & \Leftrightarrow F_\theta^{-1}(p) \leq x - \epsilon < F_n^{-1}(p) \\
    & \Leftrightarrow F_n(x) < F_\theta(x) \\
    & \Leftrightarrow \exists \delta > 0 : F_n(x) - F_\theta(x) > \delta . \\
\end{align*}
Dies gilt analog für $F_\theta^{-1}(x) - F_n^{-1}(x) > \epsilon$
\begin{align*}
    & \Rightarrow |F_n^{-1}(p) - F_\theta^{-1}(p)| > \epsilon \Leftrightarrow |F_n(x) - F_\theta(x)| > \delta . \\ 
\end{align*}
Insbesondere ist $\forall\epsilon'>0\quad\exists\delta'>0 :$
\begin{align*}
    & P_\theta^{\otimes n}(|F_n(x) - F_\theta(x)| > \delta') \Rightarrow P_\theta^{\otimes n}(|F_n^{-1}(p) - F_\theta^{-1}(p)| > \epsilon') , \\
\end{align*}
da F streng monoton wachsend und rechtsseitig stetig ist, aus dem $\epsilon - \delta$ \newline
stetigkeits Kriterium. Insgesamt also:
\begin{align*}
    & P_\theta^{\otimes n}(|F_n(x) - F_\theta(x)| > \delta) \rightarrow 0 \\
    & \Rightarrow P_\theta^{\otimes n}(|F_n^{-1}(p) - F_\theta^{-1}(p)| > \epsilon) \rightarrow 0 . \\
\end{align*}

\newpage

a) Für das Poisson-produktmodell gilt: 
\begin{align*}
    l : \Theta \rightarrow \mathbbm{R} , \theta \rightarrow ln(\rho_\theta(x_1,...,x_n)) & = \prod\limits_{i=1}^{n} e^{-\theta} \frac{\theta^{x_i}}{x_i !} \\
    & = -n\theta + \sum\limits_{i=1}^{n}x_i ln(\theta) - ln(x_i !) . \\
    l'(\theta) = -n + \sum\limits_{i=1}^{n} \frac{x_i}{\theta} \overset{!}{=} 0 \\
    \Rightarrow \theta = \frac{1}{n}\sum\limits_{i=1}^{n}x_i . \\
\end{align*}
$E[X_1]=\theta<\infty$ für alle $\theta$ da $\theta\in\Theta=(0, \infty)$ \newline
$\Rightarrow$ das empirische Mittel is konsistent auf das Produktmodell. 

\end{document}