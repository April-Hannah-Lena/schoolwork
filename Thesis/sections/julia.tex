\section{Julia}

% -------------------------------------------------------------------------------------- %

\subsection{A (Very) Brief Introduction to the Julia Language}

The julia language GitHub page decribes julia as "a high-level, high-performance dynamic 
language for technical computing" \cite*{juliarepo}. Its creators come from Matlab, Python,
Lisp among others, and they desired a langauge as syntactically simple as Python, as 
powerful for linear algebra as Matlab, and as fast as C \cite*{julia}. To achieve this,
julia is just-in-time compiled using LLVM to generate native machine code. Further, "julia uses 
multiple dispatch as a paradigm, which makes it easy to express many object-oriented and
functional programming patterns" \cite*{juliamain}. As a brief example of multiple dispatch,
consider a simple user defined type \jlinl{2dBox} (similar to the implementation in 
\texttt{GAIO.jl}):

\begin{jllisting}[language=julia, style=jlcodestyle]
    struct 2dBox
        center::Tuple{Float64,Float64}
        radius::Tuple{Float64,Float64}
    end
\end{jllisting}

Then we can create a \jlinl{2dBox} by calling \jlinl{b = 2dBox((0.0, 0.0), (1.0, 1.0))}. 
A natural question to ask is whether a vector, say \jlinl{x = [0.5, 0.6]}, lies within 
\jlinl{b}. To do this, we can \emph{overload} the function \jlinl{in} from julia base:

\begin{jllisting}[language=julia, style=jlcodestyle]
    function Base.in(x::Vector, b::2dBox)
        all( b.center .- b.radius .≤ x .≤ b.center .+ b.radius )
    end
\end{jllisting}

This function uses julia's \emph{dot-syntax} to vectorize operations, eg. \jlinl{+}
by writing \jlinl{.+}. We can now call the base function \jlinl{in} or its unicode alias
\jlinl{∈}:

\begin{jllisting}[language=julia, style=jlcodestyle]
    x = [0.5, 0.6]
    b = 2dBox((0.0, 0.0), (1.0, 1.0))

    x ∈ b           # returns true
    x .- 2 ∈ b      # returns false
    x .- 2 ∉ b      # returns true
\end{jllisting}

The above example demonstrates the syntactical simplicity of julia. 
It should be noted that despite this simplicity, julia can generate highly performant 
machine code. For further illustration, consider the algorithm for the global unstable 
manifold (see \autoref{alg:manifold}) implemented in matlab (\autoref{lst:gum:matlab}) 
and in julia (\autoref{lst:gum:julia}). The julia code is one third as long, mirrors 
the pseudocode much more closely, and still runs faster than the matlab code.

\begin{jllisting}[float, language=matlab, style=jlcodestyle, label=lst:gum:matlab, caption=Unstable manifold algorithm in matlab]
    function gum(t, f, X, depth)
    dim = t.dim;
    none = 0; ins = 2; expd = 4;             % defining flags
    nb0 = 0;  nb1 = t.count(depth);          % bookkeeping the no. of boxes
    tic; j = 1;
    t.set_flags('all', ins, depth);
    while nb1 > nb0                          % while new boxes nonempty
    t.change_flags('all', ins, expd);        % mark inserted boxes
    b = t.boxes(depth); M = size(b,2);       % get the geometry of the boxes
    flags = b(2*dim+1, :); 
    I = find(bitand(flags,expd));            % find boxes to expand
    b = b(:,I); N = size(b,2);
    S = whos('X'); l = floor(5e7/S.bytes);
    for k = 0:floor(N/l),                    % split in chunks of size l
        K = k*l+1:min((k+1)*l,N);
        c = b(1:dim,K);                      % center ...
        r = b(dim+1:2*dim,1);                % ... and radii of the boxes
        n = size(c,2); E = ones(n,1);
        P = kron(E,X)*diag(r) + ...          % sample points in all boxes
            kron(c',ones(size(X,1),1));
        t.insert(f(P)', depth, ins, none);   % map sample points, insert boxes
    end
    t.unset_flags('all', expd);              % unflag recently expanded boxes
    nb0 = nb1; nb1 = t.count(depth);
    j = j+1;
    end
\end{jllisting}

\begin{jllisting}[float, language=julia, style=jlcodestyle, label=lst:gum:julia, caption=Unstable manifold algorithm in julia]
    function unstable_set(F::BoxMap, B::BoxSet)
        B₀ = B
        B₁ = B
        while B₁ ≠ ∅
            B₁ = F(B₁)
            B₁ = B₁ \ B₀
            B₀ = B₁ ∪ B₀
        end
        return B₀
    end
\end{jllisting}

% -------------------------------------------------------------------------------------- %

\subsection{GAIO.jl}

This section is meant as an introduction to the most important concepts in the implementation of 
\texttt{GAIO.jl}. For more information see the respective docstrings, eg. by typing 
\jlinl{julia> ? Box} into the julia REPL.\\

The package is built off of three base structures \jlinl{Box, BoxPartition, BoxMap}:

\begin{jllisting}[language=julia, style=jlcodestyle]
    struct Box{N,T <: AbstractFloat}
        center::SVector{N,T}
        radius::SVector{N,T}
    end
\end{jllisting}

A Box is a half-open generalized rectangle, that is, a product of half-open intervals
$[a_1, b_1) \times [a_2, b_2) \times \ldots \times [a_d, b_d)$. We now need to represent 
a partition $\mathcal{P}$ of boxes:

\begin{jllisting}[language=julia, style=jlcodestyle]
    struct BoxPartition{N,T,I<:Integer} <: AbstractBoxPartition{Box{N,T}}
        domain::Box{N,T}
        left::SVector{N,T}
        scale::SVector{N,T}
        dims::SVector{N,I}
        dimsprod::SVector{N,I}
    end
\end{jllisting}

Instead of keeping track of positions for each individual box, we use integer indices 
for an equidistant \emph{box grid} and only store the dimensions of the grid using 
the attribute \jlinl{dims}. The attributes \jlinl{left, scale, dimsprod} are used to 
quickly calculate box indices from a given point (see \jlinl{point_to_key, key_to_box} 
in \texttt{GAIO.jl}).\\

Finally, we need a way to convert a map $f$ defined on $Q$ to a map \jlinl{F} defined on 
boxes of a \jlinl{BoxPartition}:

\begin{jllisting}[language=julia, style=jlcodestyle]
    struct SampledBoxMap{A,N,T,F,D,I} <: BoxMap
        map::F
        domain::Box{N,T}
        domain_points::D
        image_points::I
        acceleration::A
    end
\end{jllisting}

When a \jlinl{SampledBoxMap} is initialized, we require a function to calculate test 
points that are mapped by the given function $f$. This is \jlinl{domain_points}. These test 
points are then mapped forward by $f$, and the boxes which are hit become the image set.\\

Tha \jlinl{acceleration} attribute is what we will concern ourselves with for the remainder 
of the present paper. Naturally, if an increase in accuracy is required, a larger set of 
test points may be chosen. This leads to a dilemma: the more accurate we wish our 
approximation \jlinl{F} to be, the more we need to map very similar test points forward,
causing a considerable slow down for complicated dynamical systems. However, the process of
mapping each test point forward is completely independent on other test points. This means 
we do not need to perform each calculation sequentially; we can \emph{parallelize}. 

% -------------------------------------------------------------------------------------- %
