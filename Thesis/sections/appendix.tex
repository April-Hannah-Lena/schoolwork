% -------------------------------------------------------------------------------------- %

\section{Appendix}

\begin{jllisting}[float, floatplacement=hb!, language=julia, style=jlcodestyle, label=lst:boxmap, captionpos=b, caption=Function to calculate $f(\mathcal{B})$]
    function map_boxes(g::SampledBoxMap, source::BoxSet)
        P = source.partition

        # Use multithreaded iteration over the box set
        @floop for box in source
            c, r = box.center, box.radius
            domain_points = g.domain_points(c, r)
            
            # Map each test point
            for p in domain_points
                fp = g.map(p)
                hitbox = point_to_box(P, fp)

                # Skip to next iteration if `fp` lands outside `P`
                isnothing(hitbox) && continue

                # Add perturbations to `fp`
                r = hitbox.radius
                image_points = g.image_points(c, r)
                for ip in image_points
                    hit = point_to_key(P, ip)
                    isnothing(hit) && continue
                    
                    # Use @reduce syntax to initialize an empty
                    # key set and add hits during iteration
                    @reduce(image = union!(Set{keytype(P)}(), hit))
                end
            end
        end
        return BoxSet(P, image)
    end 
\end{jllisting}

\begin{jllisting}[float, floatplacement=hb!, language=julia, style=jlcodestyle, label=lst:tuple_vgather, captionpos=b, caption=Conversion function to packed tuples]
    function tuple_vgather(x::Vector{NTuple{d,T}}, s) where {d,T}
        # `x` is a vector of tuples, each of length `d` and datatype `T`
    
        n = length(x)                                   
        m = n ÷ s
        if n != m * s
            throw(DimensionMismatch("length of input not divisible by simd"))
        end
    
        # Change type interpretation of the memory of `x`
        vr = reinterpret(T, x)
    
        # Initialize the vector of packed tuples
        vo = Vector{NTuple{d,Vec{s,T}}}(undef, m)
    
        # The indices that form the first element of the first packed vector
        idx = Vec(ntuple( i -> (i-1) * d, s ))
    
        # Grab the indices of the first element of the i-th packed vector, 
        # then jump by `d*s` to grab indices of the second element, etc.
        for i in 1:m
            vo[i] = ntuple( j -> vr[idx + (i-1) * d * s + j], d )
        end                                             
        
        return vo
    end
\end{jllisting}

\begin{jllisting}[float, floatplacement=hb!, language=julia, style=jlcodestyle, label=lst:tuple_vscatter, captionpos=b, caption=Conversion back to single tuples]
    function tuple_vscatter(y::Vector{NTuple{d,Vec{s,T}}}) where {s,d,T}
        # `y` is a vector of packed tuples, tuples of SIMD Vecs

        # Initialize the unpacked vector
        vo = Vector{NTuple{d,T}}(undef, s * length(y))

        # Create a view of `vo` which is made of individual numbers of type `T` 
        # as well as a view of `y` which is made up of packed `T`-vectors
        vr = reinterpret(T, vo)
        yy = reinterpret(Vec{s,T}, y)

        # The indices that form the first element of the first packed vector
        idx = Vec(ntuple( i -> d * (i-1), s ))

        # set the values of `vr` as the permuted values of `y`
        for i in 1:d, j in 1:length(vi)
            vr[idx + (j-1) * d * s + i] = yy[i + (j-1) * d]
        end

        return vo
    end
\end{jllisting}

\begin{jllisting}[float, floatplacement=hb!, language=julia, style=jlcodestyle, label=lst:boxmap_cpu, captionpos=b, caption=Function to calculate $f(\mathcal{B})$ with CPU acceleration]
    function map_boxes(
            g::SampledBoxMap{C,N}, source::BoxSet
        ) where {simd,C<:BoxMapCPUCache{simd},N}
        
        # Preallocated temporary storage objects.
        # `temp_vec` and `temp_points` are reinterprets of
        # the same array, `temp_points` is an array of tuples.
        # `idx_base` is a SIMD Vec of indices used for `tuple_vscatter!`
        idx_base, temp_vec, temp_points = g.acceleration
        
        P = source.partition
        @floop for box in source
        
            # Grab a separate section of the
            # temporary storage for each thread
            tid  = (threadid() - 1) * simd
            idx  = idx_base + tid * N
            mapped_points = @view temp_points[tid+1:tid+simd]
            
            domain_points = g.domain_points(box.center, box.radius)
            for p in domain_points
                fp = g.map(p)

                # Scatter packed `fp` into temporary storage
                tuple_vscatter!(temp_vec, fp, idx)

                # continue as with normal `map_boxes`
                for q in mapped_points
                    hitbox = point_to_box(P, q)
                    isnothing(hitbox) && continue
                    image_points = g.image_points(q,  hitbox.radius)
                    for ip in image_points
                        hit = point_to_key(P, ip)
                        isnothing(hit) && continue
                        @reduce(image = union!(Set{keytype(P)}(), hit))
                    end
                end
            end
        end
        return BoxSet(P, image)
    end
\end{jllisting}

\begin{jllisting}[float, floatplacement=hb!, language=julia, style=jlcodestyle, label=lst:gpu_kernel, captionpos=b, caption=GPU kernel to calculate $f(\mathcal{B})$]
    function map_boxes_kernel(
            f,          # Function
            keys,       # Keys to be mapped
            points,     # "Global" test points
            out_keys,   # Array to hold mapped box indices ...
            P           # ... wrt the partition P
        )

        # Calculate the linear index of the GPU thread
        ind = (blockIdx().x - 1) * blockDim().x + threadIdx().x - 1
        stride = gridDim().x * blockDim().x
        nk, np = length(keys), length(points)
        len = nk * np - 1

        # In case `len` is too large, launch the warp repeatedly
        for i in ind : stride : len
        
            # Loop over the test points, then over the box index
            m, n = divrem(i, np) .+ 1
            p    = points[m]
            key  = keys[n]
            box  = key_to_box(P, key)
            c, r = box.center, box.radius

            # Map the test point forward
            fp   = f(@muladd p .* r .+ c)
            hit  = point_to_key(P, fp)

            # If `fp` lands outside the partition, assign it the
            # out-of-bounds index 0. This will later be discarded
            out_keys[i+1] = isnothing(hit) ? 0 : hit

        end
    end
\end{jllisting}

\begin{jllisting}[float, floatplacement=hb!, language=julia, style=jlcodestyle, label=lst:boxmap_gpu, captionpos=b, caption=Function to calculate $f(\mathcal{B})$ with GPU acceleration]
    function map_boxes(
            g::SampledBoxMap{C}, source::BoxSet
        ) where {SZ,C<:BoxMapGPUCache{SZ}}
        # BoxMapGPUCache holds a max size `SZ` of box indices 
        # which can fit in GPU memory

        # We use `Stateful` to iterate through
        # test points `SZ` points at a time
        P, keys = source.partition, Stateful(source.set)

        # Normalized test points
        points = g.domain_points(P.domain.center, P.domain.radius)
        np = length(points)

        image = Set{keytype(P)}()
        while !isnothing(keys.nextvalstate)
            stride = min(SZ, length(keys))

            # Allocate input/output indices on GPU
            in_keys = CuArray{Int32,1}(collect(Int32, take(keys, stride)))
            nk = length(in_keys)
            out_keys = CuArray{Int32,1}(undef, np * nk)

            # Compile the GPU kernel and calculate occupancy
            args = (g.map, in_keys, points, out_keys, P)
            compiled_kernel! = @cuda launch=false map_boxes_kernel!(args...)
            config  = launch_configuration(compiled_kernel!.fun)
            threads = min(np * nk, config.threads)
            blocks  = cld(np * nk, threads)

            # Launch the compiled kernel then wait until the task to complete
            compiled_kernel!(args...; threads, blocks)
            CUDA.synchronize()
            
            # Transfer output indices to CPU memory
            out_cpu = Array{Int32,1}(out_keys)
            union!(image, out_cpu)

            # Manually deallocate GPU arrays since julia has 
            # more difficulty garbage collecting GPU memory
            CUDA.unsafe_free!(in_keys); CUDA.unsafe_free!(out_keys)
        end

        # remove out-of-bounds index
        delete!(image, 0i32)

        return BoxSet(P, image)
    end
\end{jllisting}

\clearpage

\begin{table}[ht!]
    \centering
    {\footnotesize
        \begin{tabular}{| c | c c c |}
            \hline
            Number of Points  &  Time, $k=10$  & Time, $k=100$  &  Time, $k=1000$ \\
            \hline
            40  &  0.020134   &    0.16445     &     1.62278 \\
            64  &  0.0326548  &    0.2653      &     2.60109 \\
            104  &  0.059709   &    0.455421    &     4.50348 \\ 
            160  &  0.0892529  &    0.71049     &     6.76705 \\
            256  &  0.137585   &    1.20524     &    11.8306 \\
            400  &  0.217655   &    1.88588     &    18.6102 \\
            636  &  0.351438   &    3.09114     &    29.8942 \\
            1008  &  0.55545    &    4.9555      &    47.9885 \\
            1596  &  0.888532   &    7.75978     &    76.2201 \\
            2524  &  1.39837    &   12.4615      &   121.395 \\
            4000  &  2.22419    &   19.7415      &   193.073 \\
            \hline
        \end{tabular}
    }
    \caption{Times ($s$) for \autoref{alg:manifold} using no acceleration}
    \label{tb:nothing}
\end{table}

\begin{table}[ht!]
    \centering
    {\footnotesize
        \begin{tabular}{| c | c c c |}
            \hline
            Number of Points & Time, $k=10$ &  Time, $k=100$ &   Time, $k=1000$ \\
            \hline
            40  &  0.0201105   &   0.0694606    &    0.761066 \\
            64  &  0.0233355   &   0.136251     &    1.24381 \\
            104  &  0.034471    &   0.224639     &    2.3031 \\
            160  &  0.0525481   &   0.356561     &    3.62152 \\
            256  &  0.0846066   &   0.606809     &    6.04962 \\
            400  &  0.131712    &   0.962795     &    9.5455 \\
            636  &  0.211706    &   1.55601      &   16.3179 \\
            1008  &  0.341492    &   2.52285      &   27.9455 \\
            1596  &  0.540294    &   4.06358      &   43.3757 \\
            2524  &  0.866405    &   6.49313      &   68.151 \\
            4000  &  1.38411     &  10.5114       &  108.385 \\
            \hline
        \end{tabular}
    }
    \caption{Times ($s$) for \autoref{alg:manifold} using CPU acceleration}
    \label{tb:cpu}
\end{table}

\begin{table}[ht!]
    \centering
    {\footnotesize
        \begin{tabular}{| c | c c c |}
            \hline
            Number of Points & Time, $k=10$ & Time, $k=100$ & Time, $k=1000$ \\
            \hline
            40  &  0.0360844   &  0.0375707    &  0.0607846 \\
            64  &  0.125934    &  0.0353227    &  0.0780206 \\
            104  &  0.0795413   &  0.108104     &  0.125333 \\
            160  &  0.135748    &  0.147764     &  0.207934 \\
            256  &  0.185914    &  0.196439     &  0.300255 \\
            400  &  0.476413    &  0.464716     &  0.47896 \\
            636  &  0.714487    &  0.69321      &  0.863987 \\ 
            1008  &  0.774053    &  0.80273      &  1.14469 \\ 
            1596  &  1.2813      &  1.32692      &  1.8528 \\
            2524  &  2.05627     &  2.17883      &  3.10034 \\
            4000  &  3.16808     &  3.54294      &  4.86421 \\
            \hline
        \end{tabular}
    }
    \caption{Times ($s$) for \autoref{alg:manifold} using GPU acceleration}
    \label{tb:gpu}
\end{table}

\begin{table}[ht!]
    \centering
    {\footnotesize
        \begin{tabular}{| c | c c c |}
            \hline
            Number of Steps  & Time, no accel.  & Time, CPU accel. &  Time, GPU accel. \\
            \hline
            4    &    0.14474    &    0.10885    &    0.4877    \\
            6    &    0.20373    &    0.14681    &    0.48377    \\
            7    &    0.22895    &    0.14628    &    0.50268    \\
            8    &    0.25353    &    0.16037    &    0.51701    \\
            10    &    0.30581    &    0.17762    &    0.48161    \\
            13    &    0.38697    &    0.21801    &    0.47945    \\
            16    &    0.46444    &    0.26075    &    0.48565    \\
            20    &    0.56454    &    0.30852    &    0.47572    \\
            26    &    0.73231    &    0.39604    &    0.47968    \\
            32    &    0.89738    &    0.45801    &    0.50168    \\
            40    &    1.0899    &    0.55824    &    0.52061    \\
            51    &    1.38341    &    0.70655    &    0.53133    \\
            64    &    1.74327    &    0.88893    &    0.52218    \\
            80    &    2.15379    &    1.08874    &    0.49339    \\
            100    &    2.64934    &    1.35807    &    0.50099    \\
            126    &    3.41551    &    1.68225    &    0.50188    \\
            159    &    4.18167    &    2.13575    &    0.53094    \\
            200    &    5.2217    &    2.68491    &    0.52429    \\
            252    &    6.64934    &    3.3583    &    0.54523    \\
            317    &    8.2588    &    4.24002    &    0.58982    \\
            399    &    10.44096    &    5.34454    &    0.58884    \\
            502    &    13.11835    &    6.68914    &    0.58776    \\
            631    &    16.50509    &    8.41668    &    0.59624    \\
            795    &    20.72812    &    10.52048    &    0.64005    \\
            1000    &    25.78587    &    13.15393    &    0.68164    \\
            1259    &    32.51914    &    16.72895    &    0.72291    \\
            1585    &    40.89508    &    22.04528    &    0.84303    \\
            1996    &    51.75475    &    28.14989    &    0.88808    \\
            2512    &    64.75916    &    35.55423    &    0.99073    \\
            3163    &    81.7256    &    44.00314    &    1.17772    \\
            3982    &    102.74094    &    55.23752    &    1.27731    \\
            5012    &    128.53724    &    69.12566    &    1.48844    \\
            6310    &    162.24623    &    88.67913    &    1.80818    \\
            7944    &    204.59758    &    109.89934    &    2.08626    \\
            10000    &    256.9519    &    137.98999    &    2.5451    \\
            12590    &    322.56893    &    176.28813    &    3.06926    \\
            15849    &    409.50069    &    222.66347    &    3.76702    \\
            19953    &    512.10249    &    280.11334    &    4.582    \\
            25119    &    646.5692    &    351.18587    &    5.65162    \\
            31623    &    814.60274    &    443.02861    &    6.96308    \\
            39811    &    1024.4214    &    593.56447    &    8.75485    \\
            50119    &    1360.60108    &    746.596    &    10.87391    \\
            63096    &    1716.27336    &    942.11056    &    13.53368    \\
            79433    &    2160.62323    &    1181.19324    &    16.82755    \\
            100000    &    2726.68107    &    1484.66971    &    20.99817    \\
            \hline
        \end{tabular}
    }
    \caption{
        Map complexity $k$ and execution time ($s$) to calculate $f(\mathcal{B})$ from \autoref{fig:lorenz}
    }
    \label{tb:steps}
\end{table}

% -------------------------------------------------------------------------------------- %