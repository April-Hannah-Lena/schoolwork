% -------------------------------------------------------------------------------------- %

\section{Appendix}

\begin{table}[ht!]
    \centering
    {\footnotesize
        \begin{tabular}{| c | c c c |}
            \hline
            Number of Points  &  Time, $k=10$  & Time, $k=100$  &  Time, $k=1000$ \\
            \hline
            40  &  0.020134   &    0.16445     &     1.62278 \\
            64  &  0.0326548  &    0.2653      &     2.60109 \\
            104  &  0.059709   &    0.455421    &     4.50348 \\ 
            160  &  0.0892529  &    0.71049     &     6.76705 \\
            256  &  0.137585   &    1.20524     &    11.8306 \\
            400  &  0.217655   &    1.88588     &    18.6102 \\
            636  &  0.351438   &    3.09114     &    29.8942 \\
            1008  &  0.55545    &    4.9555      &    47.9885 \\
            1596  &  0.888532   &    7.75978     &    76.2201 \\
            2524  &  1.39837    &   12.4615      &   121.395 \\
            4000  &  2.22419    &   19.7415      &   193.073 \\
            \hline
        \end{tabular}
    }
    \caption{Results using no acceleration}
    \label{tb:nothing}
\end{table}

\begin{table}[ht!]
    \centering
    {\footnotesize
        \begin{tabular}{| c | c c c |}
            \hline
            Number of Points & Time, $k=10$ &  Time, $k=100$ &   Time, $k=1000$ \\
            \hline
            40  &  0.0201105   &   0.0694606    &    0.761066 \\
            64  &  0.0233355   &   0.136251     &    1.24381 \\
            104  &  0.034471    &   0.224639     &    2.3031 \\
            160  &  0.0525481   &   0.356561     &    3.62152 \\
            256  &  0.0846066   &   0.606809     &    6.04962 \\
            400  &  0.131712    &   0.962795     &    9.5455 \\
            636  &  0.211706    &   1.55601      &   16.3179 \\
            1008  &  0.341492    &   2.52285      &   27.9455 \\
            1596  &  0.540294    &   4.06358      &   43.3757 \\
            2524  &  0.866405    &   6.49313      &   68.151 \\
            4000  &  1.38411     &  10.5114       &  108.385 \\
            \hline
        \end{tabular}
    }
    \caption{Results using CPU acceleration}
    \label{tb:cpu}
\end{table}

\begin{table}[ht!]
    \centering
    {\footnotesize
        \begin{tabular}{| c | c c c |}
            \hline
            Number of Points & Time, $k=10$ & Time, $k=100$ & Time, $k=1000$ \\
            \hline
            40  &  0.0360844   &  0.0375707    &  0.0607846 \\
            64  &  0.125934    &  0.0353227    &  0.0780206 \\
            104  &  0.0795413   &  0.108104     &  0.125333 \\
            160  &  0.135748    &  0.147764     &  0.207934 \\
            256  &  0.185914    &  0.196439     &  0.300255 \\
            400  &  0.476413    &  0.464716     &  0.47896 \\
            636  &  0.714487    &  0.69321      &  0.863987 \\ 
            1008  &  0.774053    &  0.80273      &  1.14469 \\ 
            1596  &  1.2813      &  1.32692      &  1.8528 \\
            2524  &  2.05627     &  2.17883      &  3.10034 \\
            4000  &  3.16808     &  3.54294      &  4.86421 \\
            \hline
        \end{tabular}
    }
    \caption{Results using GPU acceleration}
    \label{tb:gpu}
\end{table}

\clearpage

\begin{jllisting}[float, floatplacement=hb!, language=julia, style=jlcodestyle, label=lst:boxmap, caption=Function to calculate $f(\mathcal{B})$]
    function map_boxes(g::SampledBoxMap, source::BoxSet)
        P = source.partition

        # Use multithreaded iteration over the box set
        @floop for box in source
            c, r = box.center, box.radius
            domain_points = g.domain_points(c, r)
            
            # Map each test point
            for p in domain_points
                fp = g.map(p)
                hitbox = point_to_box(P, fp)

                # Skip to next iteration if `fp` lands outside `P`
                isnothing(hitbox) && continue

                # Add perturbations to fp
                r = hitbox.radius
                image_points = g.image_points(c, r)
                for ip in image_points
                    hit = point_to_key(P, ip)
                    isnothing(hit) && continue
                    
                    # Use @reduce syntax to initialize an empty
                    # key set and add hits during iteration
                    @reduce(image = union!(Set{keytype(P)}(), hit))
                end
            end
        end
        return BoxSet(P, image)
    end 
\end{jllisting}

\begin{jllisting}[float, floatplacement=hb!, language=julia, style=jlcodestyle, label=lst:boxmap_cpu, caption=Function to calculate $f(\mathcal{B})$ with CPU acceleration]
    function map_boxes(
            g::SampledBoxMap{C,N}, source::BoxSet
        ) where {simd,C<:BoxMapCPUCache{simd},N}
        
        # Preallocated temporary storage objects.
        # `temp_vec` and `temp_points` are reinterprets of
        # the same array, `temp_points` is an array of tuples
        idx_base, temp_vec, temp_points = g.acceleration
        
        P = source.partition
        @floop for box in source
        
            # Grab a separate section of the
            # temporary storage for each thread
            tid  = (threadid() - 1) * simd
            idx  = idx_base + tid * N
            mapped_points = @view temp_points[tid+1:tid+simd]
            
            domain_points = g.domain_points(box.center, box.radius)
            for p in domain_points
                fp = g.map(p)

                # Scatter packed `fp` into temporary storage
                tuple_vscatter!(temp_vec, fp, idx)

                # continue as with normal `map_boxes`
                for q in mapped_points
                    hitbox = point_to_box(P, q)
                    isnothing(hitbox) && continue
                    image_points = g.image_points(q,  hitbox.radius)
                    for ip in image_points
                        hit = point_to_key(P, ip)
                        isnothing(hit) && continue
                        @reduce(image = union!(Set{keytype(P)}(), hit))
                    end
                end
            end
        end
        return BoxSet(P, image)
    end
\end{jllisting}

\begin{jllisting}[float, floatplacement=hb!, language=julia, style=jlcodestyle, label=lst:gpu_kernel, caption=GPU kernel to calculate $f(\mathcal{B})$]
    function map_boxes_kernel(
            f,          # Function
            keys,       # Keys to be mapped
            points,     # "Global" test points
            out_keys,   # Array to hold mapped box indices ...
            P           # ... wrt the partition P
        )

        # Calculate the index of the GPU thread
        ind = (blockIdx().x - 1) * blockDim().x + threadIdx().x - 1
        stride = gridDim().x * blockDim().x
        nk, np = length(keys), length(points)
        len = nk * np - 1

        # In case `len` is too large, launch the warp repeatedly
        for i in ind : stride : len
        
            # Loop over the test points, then the box index
            m, n = divrem(i, np) .+ 1
            p    = points[m]
            key  = keys[n]
            box  = key_to_box(P, key)
            c, r = box.center, box.radius

            # Map the point forward and record the result
            fp   = f(@muladd p .* r .+ c)
            hit  = point_to_key(P, fp)

            # If `fp` lands outside the partition, assign it the
            # out-of-bounds index 0. This will later be discarded
            out_keys[i+1] = isnothing(hit) ? 0 : hit

        end
    end
\end{jllisting}

\begin{jllisting}[float, floatplacement=hb!, language=julia, style=jlcodestyle, label=lst:boxmap_gpu, caption=Function to calculate $f(\mathcal{B})$ with GPU acceleration]
    function map_boxes(
            g::SampledBoxMap{C}, source::BoxSet
        ) where {SZ,C<:BoxMapGPUCache{SZ}}
        # BoxMapGPUCache holds a max size `SZ` of box indices 
        # which can fit in GPU memory

        # We use `Stateful` to iterate through
        # test points `SZ` points at a time
        P, keys = source.partition, Stateful(source.set)

        # Normalized test points
        points = g.domain_points(P.domain.center, P.domain.radius)
        np = length(points)

        image = Set{keytype(P)}()
        while !isnothing(keys.nextvalstate)
            stride = min(SZ, length(keys))

            # Allocate input/output indices on GPU
            in_keys = CuArray{Int32,1}(collect(Int32, take(keys, stride)))
            nk = length(in_keys)
            out_keys = CuArray{Int32,1}(undef, np * nk)

            # Compile the GPU kernel and calculate occupancy
            args = (g.map, in_keys, points, out_keys, P)
            compiled_kernel! = @cuda launch=false map_boxes_kernel!(args...)
            config  = launch_configuration(compiled_kernel!.fun)
            threads = min(np * nk, config.threads)
            blocks  = cld(np * nk, threads)

            # Launch the compiled kernel then wait until the task to complete
            compiled_kernel!(args...; threads, blocks)
            CUDA.synchronize()
            
            # Transfer output indices to CPU memory
            out_cpu = Array{Int32,1}(out_keys)
            union!(image, out_cpu)

            # Manually deallocate GPU arrays since julia has 
            # more difficulty garbage collecting GPU memory
            CUDA.unsafe_free!(in_keys); CUDA.unsafe_free!(out_keys)
        end

        # remove out-of-bounds index
        delete!(image, 0i32)

        return BoxSet(P, image)
    end
\end{jllisting}

\clearpage

\begin{table}[ht!]
    \centering
    {\footnotesize
        \begin{tabular}{| c | c c c |}
            \hline
            Number of Steps  & Time, no accel.  & Time, CPU accel. &  Time, GPU accel. \\
            \hline
            4 & 0.10733  & 0.0774621 & 0.296883 \\
            6 & 0.143625 & 0.0952406 & 0.364458 \\
            7 & 0.159921 & 0.103962  & 0.289842 \\
            8 & 0.182721 & 0.111939  & 0.294995 \\
            10 & 0.217213 & 0.130934  & 0.35477 \\
            13 & 0.278128 & 0.157542  & 0.462502 \\
            16 & 0.335391 & 0.183797  & 0.463087 \\ 
            20 & 0.40894  & 0.224452  & 0.463877 \\ 
            26 & 0.522386 & 0.279074  & 0.466465 \\
            32 & 0.619818 & 0.32773  & 0.311194 \\
            40 & 0.783703 & 0.400696 & 0.369541 \\
            51 & 0.993318 & 0.511517 & 0.365205 \\
            64 & 1.26785  & 0.608839 & 0.400862 \\
            80 & 1.554    & 0.780857 & 0.475648 \\
            100 & 1.91483  & 0.979949 & 0.336051 \\
            126 & 2.4347   & 1.19774  & 0.340938 \\
            159 & 3.05702  & 1.52709  & 0.335395 \\
            200 & 3.77057  & 1.88865  & 0.39955 \\
            252 & 4.79369  & 2.36452  & 0.436983 \\
            317 &  5.9642  &  2.97117 & 0.457834 \\
            399 &  7.50092 &  3.72863 & 0.432015 \\
            502 &  9.30319 &  4.76645 & 0.459764 \\ 
            631 & 11.6748  &  6.0041  & 0.446623 \\
            795 & 14.5952  &  7.5396  & 0.409987 \\ 
            1000 & 18.3667  &  9.63419 & 0.439264 \\
            1259 & 23.3917  & 11.9185  & 0.598132 \\ 
            1585 & 29.741   & 16.0015  & 0.62983 \\
            1996 & 36.8385  & 21.5996  & 0.715319 \\
            2512 & 47.0677  & 27.0529  & 0.669515 \\
            3163 &  58.3913 &  33.1406 & 0.922343 \\
            3982 &  74.9216 &  41.7995 & 0.873487 \\
            5012 &  92.3616 &  52.2589 & 1.24426 \\ 
            6310 & 116.477  &  64.3065 & 1.218 \\
            7944 & 145.294  &  82.0594 & 1.44954 \\
            10000 & 183.667  & 103.063  & 1.75376 \\
            12590 & 227.309  & 128.813  & 2.17254 \\
            15849 & 289.176  & 164.242  & 2.73257 \\
            19953 & 365.261  & 202.171  & 3.32561 \\
            25119 & 463.006  & 257.573  & 4.1719 \\
            31623 &  578.231 &  320.201 &  4.98578 \\
            39811 &  778.834 &  427.457 &  6.23462 \\
            50119 &  980.618 &  545.352 &  7.76651 \\
            63096 & 1237.09  &  678.236 &  9.75043 \\
            79433 & 1560.56  &  857.923 & 11.9667 \\
            100000 & 1970.96  & 1080.9   & 15.0591 \\
            \hline
        \end{tabular}
    }
    \caption{Detailed results comparing map complexity and execution time}
    \label{tb:steps}
\end{table}

% -------------------------------------------------------------------------------------- %