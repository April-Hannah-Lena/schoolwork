% -------------------------------------------------------------------------------------- %

\subsection{Relative Attractor}

The construction of a fundamental neighborhood $U$ for a global attractor $A$ is
relatively difficult, but the description $A = \bigcap_{k \geq 0} f^k(U)$ lends to 
a natural \emph{ansatz} for its approximation using a compact subdomain $Q \subset M$. \\

\begin{definition}[Realtive Global Attractor]
    Let $Q$ be compact. Then we define the \emph{attractor relative to $A$} as

    \begin{equation}
        A_Q = \bigcap\limits_{k \geq 0} f^k (Q)
    \end{equation}

\end{definition}

\begin{remark}
    It follows from the definition that the relative global attractor is a subset of the 
    global attractor.\\
\end{remark}

The idea to approximate the relative gloabl attractor is in two steps: first, we subdivide 
each of the boxes and second, discard all those boxes which do not intersect with the 
previous box set. The algorithm accepts the map $f$, the box set $\mathcal{B}$, and a 
predefined number of steps $n$. 

\begin{algorithm}
    \caption{Relative Attractor}
    \label{alg:relativeattractor}
    \begin{algorithmic}[1]
        %\Procedure{relative_attractor}{$f,\ \mathcal{B},\ n$}
        \State $\mathcal{B}_0 \gets \mathcal{B}$

        \For{$i = \left\{ 1,\ \dotsc,\ n \right\}$}
            \State $\mathcal{B}_i \gets$ \Call{subdivide}{$\,\mathcal{B}_{i-1}\,$}
            \State $\mathcal{B}_i \gets \mathcal{B}_i \cap f (\,\mathcal{B}_i\,)$
        \EndFor

        \State \Return $\mathcal{B}_n$ 

        \LComment{Optionally, the set 
        $\left\{ \mathcal{B}_0,\ \mathcal{B}_1,\ \dotsc,\ \mathcal{B}_n \right\}$
        can be returned instead}
        %\EndProcedure
    \end{algorithmic}
\end{algorithm}

\begin{remark}
    The precise technique for subdivision can be tuned depending on the situation. In 
    \texttt{GAIO.jl}, boxes are bisected evenly along one dimension 
    $k \in \left\{1,\ \dotsc,\ d\right\}$. The dimension $k$ along which to bisect is 
    cycled through during the steps. \\
\end{remark}

\begin{proposition}
    \cite*{algGAIO,subalg} Set $Q_i = \bigcup\limits_{B \in \mathcal{B}_i} B$. For all 
    $i$ we have

    \begin{enumerate}
        \item $Q_{i+1} \subset Q_i$
        \item $A_Q \subset Q_i$ 
    \end{enumerate}

    Further, if we set $Q_\infty = \bigcap\limits_{n \geq 1} Q_n$, then 
    $A_Q = Q_\infty$. In particular, this shows that $\mathcal{B}_\infty$ 
    is backward-invariant. 
\end{proposition}

