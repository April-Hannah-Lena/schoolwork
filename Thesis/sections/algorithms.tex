% -------------------------------------------------------------------------------------- %

The basic technique of all the topological algorithms is to split the compact set $Q$ into 
a partition $\mathcal{P}$ of \emph{boxes} - that is, generalized rectangles, each with 
center vector $r$ and componentwise radii $r$. The algorithms will begin with a set of boxes 
$\mathcal{B}$, and then repeatedly subdivide each box in $\mathcal{B}$ into two (or more) 
smaller boxes, examine the dynamics of the subdivided boxes, and refine the box set to 
include only the boxes we are interested in. \\

\begin{definition}[Image of a Box Set]
    For a partition $\mathcal{P}$ of $Q$ into boxes, and a subset 
    $\mathcal{B} \subset \mathcal{P}$, we will call the \emph{image of $\mathcal{B}$ under f}
    as the set of boxes which intersect with the image $f(B)$, for at least one 
    $B \in \mathcal{B}$. More precisely, it is
    
    \begin{equation}
        f (\mathcal{B}) = \left\{
            B \in \mathcal{P} \quad \vert \quad 
            \exists\ \tilde{B} \in \mathcal{B},\ x \in \tilde{B}\ :\ 
            f(x) \in B 
        \right\}
    \end{equation}


\end{definition}

\subsection{Relative Attractor}

The construction of a fundamental neighborhood $U$ for a global attractor $A$ is
relatively difficult, but the above description lends to a natural \emph{ansatz} for its 
approximation using a compact subdomain $Q \subset M$. \\

\begin{definition}[Realtive Global Attractor]
    Let $Q$ be compact. Then we define the \emph{attractor relative to $A$} as

    \begin{equation}
        A_Q = \bigcap\limits_{k \geq 0} f^k (Q)
    \end{equation}

\end{definition}

\begin{remark}
    It follows from the definition that the relative global attractor is a subset of the 
    global attractor.\\
\end{remark}

The idea to approximate the relative gloabl attractor is in two steps: first, we subdivide 
each of the boxes and second, discard all those boxes which do not intersect with the 
previous box set. The algorithm accepts the map $f$, the box set $\mathcal{B}$, and a 
predefined number of steps $n$. 

\begin{algorithm}
    \caption{Relative Attractor}
    \label{alg:relativeattractor}
    \begin{algorithmic}[1]
        %\Procedure{relative_attractor}{$f,\ \mathcal{B},\ n$}
        \State $\mathcal{B}_0 \gets \mathcal{B}$

        \For{$i = \left\{ 1,\ \dotsc,\ n \right\}$}
            \State $\mathcal{B}_i \gets$ \Call{subdivide}{$\,\mathcal{B}_{i-1}\,$}
            \State $\mathcal{B}_i \gets \mathcal{B}_i \cap f (\,\mathcal{B}_i\,)$
        \EndFor

        \State \Return $\mathcal{B}_n$ 

        \LComment{Optionally, the set 
        $\left\{ \mathcal{B}_i\ \vert\ i \in \left\{ 0,\ \dotsc,\ n \right\} \right\}$
        can be returned instead}
        %\EndProcedure
    \end{algorithmic}
\end{algorithm}

\begin{remark}
    The precise technique for subdivision can be tuned depending on the situation. In 
    \texttt{GAIO.jl} \cite*{GAIO}, boxes are bisected evenly along one dimension 
    $k \in \left\{1,\ \dotsc,\ d\right\}$. The dimension $k$ along which to bisect is 
    cycled through during the steps. \\
\end{remark}

\begin{proposition}
    \cite*{algGAIO,subalg} Set 
    $\mathcal{B}_\infty = \bigcap\limits_{n \geq 1} \bigcup\limits_{B \in \mathcal{B}_i} B$.

    \begin{enumerate}
        \item $A_Q \subset \bigcup\limits_{B \in \mathcal{B}_i} B$ for all 
        $i \in \left\{ 0,\ \dotsc,\ n \right\}$
        \item $A_Q = \mathcal{B}_\infty$
    \end{enumerate}

    In particular, this shows that $\mathcal{B}_\infty$ is backward-invariant. 
\end{proposition}

