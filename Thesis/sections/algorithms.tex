% -------------------------------------------------------------------------------------- %

\subsection{Relative Attractor}

The construction of a fundamental neighborhood $U$ for a global attractor $A$ is
relatively difficult, but the description \ref{eq:globattr} lends to 
a natural \emph{ansatz} for its approximation using a compact subdomain $Q \subset M$. \\

\begin{definition}[Realtive Global Attractor]
    Let $Q$ be compact. Then we define the \emph{attractor relative to $A$} as

    \begin{equation}
        A_Q = \bigcap\limits_{k \geq 0} f^k (Q)
    \end{equation}

\end{definition}

\begin{remark}
    It follows from the definition that the relative global attractor is a subset of the 
    global attractor.\\
\end{remark}

The idea to approximate the relative gloabl attractor is in two steps: first, we subdivide 
each of the boxes and second, discard all those boxes which do not intersect with the 
previous box set. The algorithm requires a map $f$, a box set $\mathcal{B}$, and a 
predefined number of steps $n$. 

\begin{algorithm}
    \caption{Relative Attractor}
    \label{alg:relativeattractor}

    \begin{algorithmic}[1]
        %\Procedure{relative_attractor}{$f,\ \mathcal{B},\ n$}
        \State $\mathcal{B}_0 \gets \mathcal{B}$

        \For{$i = \left\{ 1,\ \dotsc,\ n \right\}$}
            \State $\mathcal{B}_i \gets$ \Call{subdivide}{$\,\mathcal{B}_{i-1}\,$}
            \State $\mathcal{B}_i \gets \mathcal{B}_i\, \cap\, f (\,\mathcal{B}_i\,)$
        \EndFor

        \State \Return $\mathcal{B}_n$ 
        %\EndProcedure
    \end{algorithmic}
\end{algorithm}

\begin{remark}\
    \begin{itemize}
        \item Optionally, the set 
        $\left\{ \mathcal{B}_0,\ \mathcal{B}_1,\ \dotsc,\ \mathcal{B}_n \right\}$
        can be returned instead. This will be true for all algorithms of this type.
        \item The precise technique for subdivision can be tuned depending on the situation. 
        In \texttt{GAIO.jl}, boxes are bisected evenly along one dimension 
        $k \in \left\{1,\ \dotsc,\ d\right\}$. The dimension $k$ along which to bisect is 
        cycled through during the steps. \\
    \end{itemize}
\end{remark}

\begin{proposition}
    \cite*{algGAIO,subalg} Set $Q_i = \bigcup\limits_{B \in \mathcal{B}_i} B$. For all 
    $i$ we have

    \begin{enumerate}
        \item $Q_{i+1} \subset Q_i$
        \item $A_Q \subset Q_i$ 
    \end{enumerate}

    Further, if we set $Q_\infty = \bigcap\limits_{n \geq 1} Q_n$, then 
    $A_Q = Q_\infty$. In particular, this shows that $\mathcal{B}_\infty$ 
    is backward-invariant. 
\end{proposition}

% -------------------------------------------------------------------------------------- %

\subsection{Unstable Manifold}

From the definition of the local unstable manifold $W^u(\bar{x}, U)$ we see that the 
relative gloabl attractor $R_Q(f)$ contains the local unstable manifold, and, provided the 
set $Q$ is sufficiently small, $W^u(\bar{x}, U)$ coincides with $R_Q(f)$. For further 
details, see \cite*{manifold, geodynbook}. Using this knowledge, we can approximate the 
global unstable manifold $W^u(\bar{x})$: 

\begin{enumerate}
    \item first, we perform an \emph{initialiation step}: replace the calculation of 
    the local unstable manifold with the calculation of the relative
    attractor for a small set $Q'$ surrounding the fixed point $\bar{x}$, using algorithm
    \ref{alg:relativeattractor}. 
    \item Second, we repeat the following \emph{continuation step}: map 
    the current box set forward one iteration, and note any new boxes which are hit. These 
    new boxes get added to the box set. Repeat until there are no new boxes
    added to the set. 
\end{enumerate}

\begin{algorithm}
    \caption{Continuation Step}
    \label{alg:manifold}

    \begin{algorithmic}[1]
        \State $\mathcal{B}_0 \gets \mathcal{B}$
        \State $\mathcal{B}_1 \gets \mathcal{B}$

        \While{$\mathcal{B}_0 \neq \emptyset$}
            \State $\mathcal{B}_0 \gets f(\,\mathcal{B}_0\,)$
            \State $\mathcal{B}_0 \gets \mathcal{B}_0\, \setminus\, \mathcal{B}_1$
            \State $\mathcal{B}_1 \gets \mathcal{B}_1\, \cup\, \mathcal{B}_0$
        \EndWhile

        \State \Return $\mathcal{B}_1$
    \end{algorithmic}
\end{algorithm}

\begin{proposition}
    \cite*{manifold} The algorithm in general cannot guarantee covering of the 
    entire unstable manifold, nor can it guarantee covering of the entirety of 
    $W^u(\bar{x}) \cap Q$. This is because $W^u(\bar{x})$ could in theory exit $Q$, 
    but return at another point. The algorithm can however guarantee covering 
    of the connected component of $W^u(\bar{x}) \cap Q$ which contains $\bar{x}$.
\end{proposition}
    
% -------------------------------------------------------------------------------------- %

\subsection{Chain Recurrent Set}

The algorithm to compute the chain recurrent set is first due to \cite*{chain}. The idea 
is to construct a directed graph whose vertices are the box set $\mathcal{B}$, and for 
which edges are drawn from $B_1$ to $B_2$ if $f$ maps any part of $B_1$ into $B_2$. 
Call this graph $G$, and call $\bar{d}$ the maximum \emph{diameter} of a box in our 
partition, ie 
$\bar{d} = \max_{\substack{B\, \in\, \mathcal{P},\\ x,\, y\, \in\, B}} d(x, y)$.\\ 

We can now ask for a subset of the vertices, for which each vertex is part of a cycle. 
Then this subset contains all points for which there exists a $\bar{d}$-pseudoperiodic orbit.

\begin{definition}[Strongly Connected]
    For a directed graph $G = (V, E)$, a subset $H \subset V$ of vertices is called 
    \emph{strongly connected} if for all $u,\, v\, in H$ there exist paths in both 
    directions between $v$ and $u$. The strongly connected component 
    \textproc{scc}$(G,\, u)$ is the strongly connected subgraph which includes $v$.
\end{definition}