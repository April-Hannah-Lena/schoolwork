% -------------------------------------------------------------------------------------- %

\subsection{Motivation}\label{sec:mot}

Our goal is to investigate the qualitative, long-term behavior of systems in which a given 
function describes the trajectory of a point in an ambient space. Such dynamical systems 
are used in modelling physcial phenomena, economic forecasting, differential equations, etc. 
We wish to construct topological \emph{closed covers} of sets which describe the infinite 
dynamics of some portions of the system, as well as statistical \emph{invariant measures} 
which describe much larger sets in the space, but with less information. \\

The basic technique of all the topological algorithms is to split a compact set $Q$ into 
a partition $\mathcal{P}$ of \emph{boxes} - that is, generalized rectangles, each with 
center vector $c$ and componentwise radii $r$. The algorithms will begin with a set of boxes 
$\mathcal{B}$, and then repeatedly subdivide each box in $\mathcal{B}$ into two (or more) 
smaller boxes, examine the dynamics of the subdivided boxes, and refine the box set to 
include only the boxes we are interested in. \\

The algorithms described in the present paper have been previously implemented in the 
statistical programming language \texttt{matlab} \cite*{matlab}, but is now being fully refactored and
reimplemented in the open-source, composable language \texttt{julia} \cite*{julia}. The reason for this 
change is julia's high-level abstraction capabilities, just-in-time compilation, and 
in-built set-theoretical functions, which create short, elegant code which is nonetheless 
more performant. Source code for \texttt{GAIO} in \texttt{matlab} and \texttt{julia} and be 
found in \cite*{oldGAIO} and \cite*{GAIO}, respectively.

\subsection{Definitions}

This section should be treated as an index of definitions, to be referred back to as 
necessary during reading. In the following, we assume $M$ is a compact or a smooth manifold 
in $\mathbb{R}^d$, endowed with a metric $d$, and the 
map $f: M \to M$ is at least $\mathcal{C}^0$. Our setting is a 
\emph{discrete, autonomous dynamical system}, that is, a system of the form:

\begin{equation}
    x_{k+1} = f(x_k), \quad k = 0,\ 1,\ 2,\ \dotsc
\end{equation}

A continuous dynamical system $\dot{x} = F(x)$ can be \emph{discretized} by, for example,
considering the \emph{Poincaré time-$t$ map} over some $d-1$ dimensional hyperplane,
or by setting one "step" of the system as integrating $F$ for a set time $t$. \\

We begin by giving a set of topological definitions of sets we wish to approximate.

\begin{definition}[(Forward-, Backward-) Invariant]
    \cite*{algGAIO} A set $A$ is called \emph{forward-invariant} if $f(A) \subset A$, 
    \emph{backward-invariant} if $f^{-1}(A) \subset A$, and \emph{invariant} if it is
    both forward- and backward-invariant.
\end{definition}

\begin{definition}[Attracting Set]
    \cite*{subalg} An invariant set $A$ is called \emph{attracting} with \emph{fundamental neighborhood}
    $U$ if for every open set $V \supset A$ there is an $N \in \mathbb{N}$ such that the tail 
    $\bigcup_{k \geq N} f^k(U)$ lies entirely within $A$. The attracting 
    set is also called \emph{global} if the \emph{basin of attraction} 
    
    \begin{equation}
        B(A) = \bigcap\limits_{k \geq 0} f^{-k}(U)
    \end{equation}

    is the whole of $\mathbb{R}^n$.
\end{definition}

The basin of attraction acts in some sense as the set for which all points eventually arrive
in $A$. Since the map $f$ is smooth, then the closure $\bar{A}$ is invariant too. With continuity 
it becomes clear that

\begin{equation}
    A = \bigcap\limits_{k \geq 0} f^k(U).
\end{equation}

The global attractor is maximal in the sense that it contains all backward-invariant 
sets within the system. In particular, it contains local unstable manifolds.

\begin{definition}[Stable and Unstable Manifolds]
    \cite*{dynbook} Let $\bar{x}$ be a fixed point of the diffeomorphism $f$, and $U$ a neighborhood of $x$. 
    Then the \emph{local unstable manifold} is given by

    \begin{equation}
        W^u(\bar{x}, U) = \left\{x \in U\ \vert\ \lim\limits_{k \to \infty} 
        d(f^{-k}(x),\ \bar{x}) = 0\ \text{and}\ f^{-k}(x) \in U\ \forall\, k \geq 0\right\}.
    \end{equation}

    The \emph{global unstable manifold} is given by

    \begin{equation}
        W^u(\bar{x}) = \bigcup\limits_{k \geq 0} f^k(\,W^u(\bar{x}, U)\,).
    \end{equation}

    The dual definition of the \emph{(local) stable manifold} is obtained by reversing the 
    sign of $k$ in the above equations.
\end{definition}

\begin{figure}[ht]
    \ctikzfig{unstablemanifold}
    \caption{\cite*{dynskript} Stable and unstable manifolds, local and global}
    \label{fig:manifold}    
\end{figure}

\begin{definition}[Pseudoperiodic]
    \cite*{dynbook} Let $n\in \mathbb{N}$. A set 
    $\left\{ x_k\ \vert\ k \in \left\{ 0,\ \dotsc ,\ n \right\} \right\}$ 
    is called \emph{$\epsilon$-pseudoperiodic} if for any
    $k, \quad d(x_{k\, \text{mod}\, n},\ x_{k + 1\, \text{mod}\, n}) < \epsilon$.
\end{definition}

As the name suggests, an $\epsilon$-pseudoperiodic orbit is "almost" periodic inthe sense 
that it represents a "small" perturbation of a theoretically periodic orbit. In practice,
such direct orbits may not be known, but it will preresent a naturally useful definition
in our approximations.

\begin{figure}[ht]
    \ctikzfig{pseudoperiodic}
    \caption{\cite*{dynbook} A $0.1$-pseudoperiodic orbit of the map $f(x,y) = (y,\ 0.05\, (1 - x^2)\, y - x)$}
    \label{fig:pseudoperiodic}
\end{figure}

\begin{definition}[Chain Recurrent]
    \cite*{dynbook} The point $\bar{x} \in M$ is called \emph{chain recurrent} if for any $\epsilon > 0$ 
    there exists an $\epsilon$-pseudoperiodic orbit. The \emph{chain recurrent set} $R_M(f)$ 
    is the set of all chain recurrent points in $M$.
\end{definition}

As shown in \cite*{algGAIO} we have the inclusion $R_M(f) \subset \bigcap_{k \geq 0} f^k(M)$,
which also shows that $R_M(f)$ is an invariant set. \\

% -------------------------------------------------------------------------------------- %
We continue with a set of measure-theoretical definitions for types of measures we wish to 
approximate. \\

Since our goal is to partition the manifold into a finite set of boxes, we must accept some
amount of "uncertainty" in how our sets look, and how \emph{exactly} $f$ maps such a set.
We describe this noise using a stochastic transition function. 

\begin{definition}[Transition Function]
    \cite*{attr} Let $\mathfrak{B}$ be the Borel $\sigma$-agebra on $M$. A function 
    $p : M \times \mathfrak{B} \to [0,1]$ is called \emph{transition function} if

    \begin{enumerate}
        \item $q(\cdot, A) : M \to [0,1]$ is measurable for all $A \in \mathfrak{B}$,
        \item $q(x, \cdot) : \mathfrak{B} \to [0,1]$ is a probability measure for all $x \in M$.
    \end{enumerate}

\end{definition}

\begin{example}

    \begin{itemize}
        \item \cite*{attr} We can model the deterministic system using the dirac measure \\
        $p(x,\, A) = \delta_{f(x)}(A)$. 
        \item The approximate box version of the system can be modelled as using a uniform 
        probability density: Let $\mathcal{P}$ be a partition of $Q$ into equally sized,
        disjoint boxes (think of a checkerboard). Then for a point $x$, find the box 
        $B \in \mathcal{P}$ with which contains $x$ and map it forward to $f(B)$. Finally, let 
        $\mathcal{B} \subset \mathcal{P}$ be a cover of $f(B)$.

        \begin{equation}
            p(x,\, A) = \frac{
                \mathcal{L} \left( A\ \cap\ \bigcup_{B \in \mathcal{B}} B \right)
            }{
                \mathcal{L} \left(\bigcup_{B \in \mathcal{B}} B \right)
            },
        \end{equation}

        where $\mathcal{L}$ represents the $d$-dimensional Lebesque measure.
    
    \end{itemize}

\end{example}

\begin{definition}[Perron-Frobenius Operator, Invariant Measure]
    \cite*{attr} Let $p$ be a stochastic transition function, and $\mu$ a measure on $M$. 
    We define the \emph{Perron-Frobenius operator} as
    
    \begin{equation}
        (P\mu)(A) = \int p(x,\, A)\ d\mu (x)
    \end{equation}
    
    A measure $\mu$ is called \emph{invariant} if it is a fixed point of $P$.
\end{definition}

\begin{remark}
    The Perron-Frobenius operator is often also called \emph{transfer operator}.
\end{remark}

\begin{example} 
    \cite*{attr} We calculate

    \begin{equation}
        (P\mu)(A) = \int \delta_{f(x)} (A)\ d\mu (x) 
        = \int \chi_A (f(x))\ d\mu (x) = \mu \circ f^{-1}(A).
    \end{equation}

    In this case $P$ simply becomes the \emph{pushforward operator}.
\end{example}

An invariant measure can be used to understand the global behavior of a dynamical 
system, with more $\mu$-mass assigned to regions which are visited frequently over long
trajectories, and less $\mu$-mass to regions visited less frequently. \\

Our "noisy" approximated system poses the benefit that while deterministic dynamical systems 
generally support the existence of multiple invariant measures, the stochastic system will 
(under mild assumptions) have a unique invariant measure, as shown in \emph{WHERE IS THIS SHOWN?}. \\

A fixed point - or eigenmeasure with eigenvalue $1$ - is not the only object of interest 
when considering the operator $P$. Suppose instead we have a deterministic dynamical system
and a finite (complex valued) measure with $P \nu = \lambda \nu$ for a $\lambda = -1$. Then, 
using finiteness and borel measurability, we can find a partition of M in two disjoint subsets 
$A_1, A_2$ such that $\nu (A_1) = - \nu (A_2)$. In particular, this implies that $f$ maps 
$A_1$ to $A_2$, and $A_2$ to $A_1$ (since $P^2 \nu = \nu$). This paritition forms a 
\emph{two-cycle}. \\

Finally, we set some notation for convenience.

\begin{definition}[Image of a Box Set]
    For a partition $\mathcal{P}$ of $Q$ into boxes, and a subset 
    $\mathcal{B} \subset \mathcal{P}$, we will call the \emph{image of $\mathcal{B}$ under f} 
    the set of boxes which intersect with the image $f(B)$, for at least one 
    $B \in \mathcal{B}$. More precisely, it is 
    
    \begin{equation}
        f (\mathcal{B}) = \left\{
            R \in \mathcal{P} \quad \vert \quad 
            f^{-1} (R)\, \cap\, \bigcup\limits_{B \in \mathcal{B}} B \neq \emptyset
        \right\}.
    \end{equation}

\end{definition}

\begin{theorem}[Image of a Box Set]
    $f(\mathcal{B})$ is the inclusion-minimal cover of $f(\bigcup_{B \in \mathcal{B}} B)$
    with boxes from $\mathcal{P}$.
\end{theorem}

\begin{proof}
    We have the equivalent characterisation

    \begin{equation}
        f(\mathcal{B}) = \left\{
            R \in \mathcal{P} \quad \vert \quad
            \exists\ B \in \mathcal{B}\ \text{and}\ x \in B\ :\ f(x) \in R
        \right\}.
    \end{equation}

    Hence if we remove one box $R$ from $f(\mathcal{B})$, then there exists an 
    $x \in \bigcup_{B \in \mathcal{B}} B$ which maps outside of the created box set.

\end{proof}

\begin{figure}[ht]
    \ctikzfig{boximage}
    \caption{Image of the simple box set $\mathcal{B} = \left\{ B \right\}$}
    \label{fig:boximage}
\end{figure}