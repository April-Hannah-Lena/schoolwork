\section{Dynamical Systems}

% -------------------------------------------------------------------------------------- %

\subsection{Motivation}\label{sec:mot}

Our goal is to investigate the qualitative, long-term behavior of systems in which a given 
function describes the trajectory of a point in an ambient space. Such dynamical systems 
are used in modelling physcial phenomena, economic forecasting, differential equations, etc. 
We wish to construct topological \emph{covers} of sets which describe the infinite-time 
dynamics of some portions of the system, as well as statistical \emph{invariant measures} 
which describe much larger sets in the space, but with less information. \\

The basic technique of all the topological algorithms is to split a compact set $Q$ into 
a partition $\mathcal{P}$ of \emph{boxes} - that is, generalized rectangles
$[c_1 - r_1,\ c_1 + r_1) \times \ldots \times [c_d - r_d,\ c_d + r_d)$, with 
center vector $c \in \mathbb{R}^d$ and componentwise radii $r \in \mathbb{R}^d$. 
The algorithms will begin with a set of boxes 
$\mathcal{B}$, and then repeatedly subdivide each box in $\mathcal{B}$ into two (or more) 
smaller boxes, examine the dynamics of the subdivided boxes, and refine the box set to 
include only the boxes we are interested in. \\

The algorithms described in the present paper have been previously implemented in the 
statistical programming language matlab \cite*{matlab}, but is now being fully refactored and
reimplemented in the open-source, composable language julia \cite*{julia}. The reason for this 
change is julia's high-level abstraction capabilities, just-in-time compilation, and 
in-built set-theoretical functions, which create short, elegant code which is nonetheless 
more performant. Source code for GAIO in matlab and julia and be 
found in \cite*{oldGAIO} and \cite*{GAIO}, respectively. \\

In the following, we assume $M$ is a compact or a smooth manifold 
in $\mathbb{R}^d$, endowed with a metric $d$, and the 
map $f: M \to M$ is at least a homeomorphism. We further assume that $P$ is a partition 
of a compact set $Q \subset M$ into (up to a Lebesque-null set) disjoint boxes, and 
$\mathcal{B} \subset \mathcal{P}$ is a subset of boxes. Our setting is a 
\emph{discrete, autonomous dynamical system}, that is, a system of the form:

\begin{equation}
    x_{k+1} = f(x_k), \quad k = 0,\ 1,\ 2,\ \dotsc
\end{equation}

A continuous dynamical system $\dot{x} = F(x)$ can be \emph{discretized} by, for example,
considering the \emph{Poincaré time-$t$ map} over some $d-1$ dimensional hyperplane,
or by setting one "step" of the system as integrating $F$ for a set time $t$. 

% -------------------------------------------------------------------------------------- %

\subsection{Some Notation}

\begin{definition}[Image of a Box Set]
    We will call the \emph{image of $\mathcal{B}$ under f} 
    the set of boxes which intersect with the image $f(B)$, for at least one 
    $B \in \mathcal{B}$. More precisely, it is 
    
    \begin{equation}
        f (\mathcal{B}) = \left\{
            R \in \mathcal{P} \quad \vert \quad 
            f^{-1} (R)\, \cap\, \bigcup\limits_{B \in \mathcal{B}} B \neq \emptyset
        \right\}.
    \end{equation}

\end{definition}

\begin{theorem}
    If $\mathcal{P}$ is pairwise disjoint, then 
    $f(\mathcal{B})$ is the inclusion-minimal cover of $f(\bigcup_{B \in \mathcal{B}} B)$
    with boxes from $\mathcal{P}$.
\end{theorem}

\begin{proof}
    We have the equivalent characterisation

    \begin{equation}
        f(\mathcal{B}) = \left\{
            R \in \mathcal{P} \quad \vert \quad
            \exists\ B \in \mathcal{B}\ \text{and}\ x \in B\ :\ f(x) \in R
        \right\}.
    \end{equation}

    Hence if we take one box $R \in f(\mathcal{B})$, then there exists an 
    $x \in \bigcup_{B \in \mathcal{B}} B$ which maps into $R$. Since 
    $\mathcal{P}$ is pairwise disjoint, $f(\mathcal{B}) \setminus \left\{ R \right\}$ 
    no longer covers $f(x)$. 
\end{proof}

\begin{figure}[ht]
    \ctikzfig{boximage}
    \caption{Image of the simple box set $\mathcal{B} = \left\{ B \right\}$}
    \label{fig:boximage}
\end{figure}

\begin{definition}[Diameter of a Box Set]
    Denote by $d(\mathcal{B})$ the maximum \emph{diameter} of a box $B \in \mathcal{B}$, ie 
    $d(\mathcal{B}) = \max_{\substack{B\, \in\, \mathcal{B},\\ x,\, y\, \in\, B}} d(x, y)$. 
\end{definition}

% -------------------------------------------------------------------------------------- %

\subsection{Topological Definitions}

\begin{definition}[(Forward-, Backward-) Invariant]
    \cite*{algGAIO} A set $A$ is called \emph{forward-invariant} if $f(A) \subset A$, 
    \emph{backward-invariant} if $f^{-1}(A) \subset A$, and \emph{invariant} if it is
    both forward- and backward-invariant.
\end{definition}

\begin{definition}[Attracting Set]
    \cite*{subalg} An invariant set $A$ is called \emph{attracting} with \emph{fundamental neighborhood}
    $U$ if for every open set $V \supset A$ there is an $N \in \mathbb{N}$ such that the tail 
    $\bigcup_{k \geq N} f^k(U)$ lies entirely within $V$. The attracting 
    set is also called \emph{global} if the \emph{basin of attraction} 
    
    \begin{equation}\label{eq:globattr}
        B(A) = \bigcap\limits_{k \geq 0} f^{-k}(U)
    \end{equation}

    is the whole of $\mathbb{R}^n$.
\end{definition}

The basin of attraction acts in some sense as the set for which all points approach
in $A$. If the map $f$ is smooth, then the closure $\bar{A}$ is invariant too. 
If $A$ is closed, it becomes clear with continuity of $f$ that 

\begin{equation}
    A = \bigcap\limits_{k \geq 0} f^k(U).
\end{equation}

The global attractor is maximal in the sense that it contains all backward-invariant 
sets within the system. In particular, it contains local unstable manifolds.

\begin{definition}[Stable and Unstable Manifolds]
    \cite*{dynbook} Let $\bar{x}$ be a fixed point of the diffeomorphism $f$, and $U$ a neighborhood of $x$. 
    Then the \emph{local unstable manifold} is given by

    \begin{equation}
        W^u(\bar{x}, U) = \left\{x \in U\ \vert\ \lim\limits_{k \to \infty} 
        d(f^{-k}(x),\ \bar{x}) = 0\ \text{and}\ f^{-k}(x) \in U\ \forall\, k \geq 0\right\}.
    \end{equation}

    The \emph{global unstable manifold} is given by

    \begin{equation}
        W^u(\bar{x}) = \bigcup\limits_{k \geq 0} f^k(\,W^u(\bar{x}, U)\,).
    \end{equation}

    The dual definition of the \emph{(local) stable manifold} is obtained by reversing the 
    sign of $k$ in the above equations.
\end{definition}

\begin{figure}[ht]
    \ctikzfig{unstablemanifold}
    \caption{\cite*{dynskript} Stable and unstable manifolds, local and global}
    \label{fig:manifold}    
\end{figure}

\begin{definition}[Pseudoperiodic]
    \cite*{dynbook} Let $n\in \mathbb{N}$. A set 
    $\left\{ x_k\ \vert\ k \in \left\{ 0,\ \dotsc ,\ n \right\} \right\}$ 
    is called \emph{$\epsilon$-pseudoperiodic} if for any
    $k,\,\ d(f(x_{k\, \text{mod}\, n}),\ x_{k + 1\, \text{mod}\, n}) < \epsilon$.
\end{definition}

As the name suggests, an $\epsilon$-pseudoperiodic orbit is "almost" periodic in the sense 
that it represents a "small" perturbation of a theoretically periodic orbit. In practice,
such pseudoperiodic orbits will not be known a-priori, but it will represent a naturally 
useful definition in our approximations.

\begin{figure}[ht]
    \ctikzfig{pseudoperiodic}
    \caption{
        \cite*{dynbook} A $0.1$-pseudoperiodic orbit of the map $f(x,y) = (y,\ 0.05\, (1 - x^2)\, y - x)$. 
        One can verify that $f(2, 0) = (0.1, -2),$ $f(0, -2) = (-2, -0.1),$ 
        $f(-2, 0) = (-0.1, 2),$ $f(0, 2) = (2, 0.1).$
    }
    \label{fig:pseudoperiodic}
\end{figure}

\begin{definition}[Chain Recurrent]
    \cite*{dynbook} The point $\bar{x} \in M$ is called \emph{chain recurrent} if for any 
    $\epsilon > 0$ there exists an $\epsilon$-pseudoperiodic orbit containing $\bar{x}$. 
    The \emph{chain recurrent set} $R_M$ is the set of all chain recurrent points in $M$.
\end{definition}

% -------------------------------------------------------------------------------------- %

\subsection{Stochastic Definitions}

Since our goal is to partition the manifold into a finite set of boxes, we must accept some
amount of "uncertainty" in how our sets look, and how \emph{exactly} $f$ maps such a set.
We describe this noise using a stochastic transition function. 

\begin{definition}[Transition Function]
    \cite*{attr} Let $\mathcal{A}$ be a $\sigma$-algebra on $M$. A function 
    $p : M \times \mathcal{A} \to [0,1]$ is called \emph{transition function} if

    \begin{enumerate}
        \item $p(\cdot, A) : M \to [0,1]$ is measurable for all $A \in \mathcal{A}$,
        \item $p(x, \cdot) : \mathcal{A} \to [0,1]$ is a probability measure for all $x \in M$.
    \end{enumerate}

\end{definition}

\begin{example}\
    
    \begin{itemize}
        \item\label{ex:q} \cite*{attr} We can model the deterministic system using the Dirac 
        measure $p(x,\, A) = \delta_{f(x)}(A)$. 
        \item The approximate box version of the system can be modelled as using a uniform 
        probability density: for a point $x$ let 
        $\mathcal{B}(x) = \left\{ B \in \mathcal{B}\ \vert\ x \in B \right\}$ be the 
        (singleton) box set containing $x$. The define $p$ as 

        \begin{equation}
            p(x, A) = \frac{
                \mathcal{L} (\ A \cap f(\mathcal{B}(x))\ )
            }{
                \mathcal{L} (\ f(\mathcal{B}(x))\ )
            }
        \end{equation}

        where $\mathcal{L}$ represents the $d$-dimensional Lebesque measure.
    
    \end{itemize}

\end{example}

\begin{definition}[Perron-Frobenius Operator, Invariant Measure]
    \cite*{attr} Let $p$ be a transition function, and $\mu$ a measure on $M$. 
    We define the \emph{Perron-Frobenius operator} as
    
    \begin{equation}
        (P\mu)(A) = \int p(x,\, A)\ d\mu (x)
    \end{equation}
    
    A measure $\mu$ is called \emph{invariant} if it is a fixed point of $P$.
\end{definition}

\begin{remark}
    The Perron-Frobenius operator is often also called \emph{transfer operator}.
\end{remark}

\begin{example} 
    \label{ex:pushforward}\cite*{attr} We calculate using the function from Ex. \ref{ex:q}

    \begin{equation}
        (P\mu)(A) = \int \delta_{f(x)} (A)\ d\mu (x) 
        = \int \chi_A (f(x))\ d\mu (x) = \mu \circ f^{-1}(A).
    \end{equation}

\end{example}

An invariant measure can be used to understand the global behavior of a dynamical 
system, with more $\mu$-mass assigned to regions which are visited frequently over long
trajectories, and less $\mu$-mass to regions visited less frequently. \\

Our "noisy" approximated system poses the benefit that while deterministic dynamical systems 
generally support the existence of multiple invariant measures, the stochastic system will 
(if the transition function has a strictly positive density for each $x \in M$) have a unique 
invariant measure, as shown in \cite*{lasota}. \\

\begin{definition}[Almost Invariant]
    \cite*{attr}
    A set $A \subset M$ is called \emph{$\delta$-almost invariant} with respect to $\mu$ if 
    $\mu (A) \neq 0$ and 

    \begin{equation}
        \int_A p(x, A)\ d\mu (x) = \delta \mu (A). 
    \end{equation}

\end{definition}

\begin{example}
    \cite*{attr}
    Consider again the deterministic case $p(x, \cdot) = \delta_{f(x)}$. Then using the same 
    calcuation as Ex. \ref{ex:pushforward} we get 

    \begin{equation}
        \delta = \frac{1}{\mu (A)} \int_A p(x, A)\ d\mu (x) 
        = \frac{\mu (A \cap f^{-1} (A))}{\mu (A)}. 
    \end{equation}

\end{example}

\begin{theorem}[Hahn-decomposition]
    \cite*{funcana}
    Let $(M, \mathcal{A}, \mu)$ be a measure space on $M$, where $\mu$ is any signed 
    measure. Then there exists $E^+ \in \mathcal{A}$ with 

    \begin{equation}
        \mu (E \cap E^+) \geq 0, \quad \mu (E \setminus E^+) \leq 0 \quad
        \text{for all } E \in \mathcal{A}\,.
    \end{equation}

    In particular $\mu (E^+) \geq 0$ and $\mu (M \setminus E^+) \leq 0$.
\end{theorem}

A fixed point - or eigenmeasure with eigenvalue $1$ - is not the only object of interest 
when considering the operator $P$. Suppose we have a
finite (potentially complex valued) measure with $P \nu = \lambda \nu$. Then
we must have 

\begin{equation}
    \lambda \nu (M) = P \nu (M) = \int p(x, M)\ d\nu (x) = \int 1\ d\nu (x) = \nu (M).
\end{equation}

Hence either $\lambda = 1$ or $\nu (M) = 0$. Now assume $\lambda \neq 0$, and that $\nu$ 
is scaled such that $| \nu |$ is a probability measure. Then by the Hahn decomposition we 
have two sets $E^+$ and $E^- = M \setminus E^+$ which partition $M$ into positive and 
negative "regions" of $M$. From $\sigma$-additivity we have

\begin{equation}
    0 = \nu (M) = \nu (E^+ \cup E^-) = \nu (E^+) + \nu (E^-). 
\end{equation}

Hence $\nu (E^+) = - \nu (E^-)$, which means $| \nu | (E^+) = | \nu | (E^-) = \frac{1}{2}$. \\

\begin{theorem}[Almost Invariant Decomposition]
    \label{thm:almost}
    \cite*{attr}
    Suppose $\nu$ is scaled such that $| \nu |$ is a probability measure, and 
    $\nu (E^+) = \frac{1}{2}$. Then $\nu = | \nu |$ over $E^+$ and 

    \begin{equation}
        \delta + \sigma = \lambda + 1
    \end{equation}

    if $E^+$ is $\delta$-almost invariant and $E^- = M \setminus E^+$ is $\sigma$-almost 
    invariant.
\end{theorem}

\begin{example}
    \label{ex:almost}
    Finally, we can consider the deterministic case with $\lambda \approx -1$. Then the sets 
    $E^+$ and $E^-$ satisfy
    
    \begin{gather}
        \delta = \nu (E^+ \cap f^{-1} (E^+))\, /\, \nu (E^+)\ \geq 0, \\
        \sigma = \nu (E^- \cap f^{-1} (E^-))\, /\, \nu (E^-)\ \geq 0, \\
        \delta + \sigma = \lambda + 1 \approx 0.
    \end{gather}
    
    This implies that nearly all the $\nu$-mass of $E^+$ gets transported to $E^-$ by $f$, 
    and vice versa. We call this an \emph{almost invariant two-cycle}.     
\end{example} 

%\newpage

% -------------------------------------------------------------------------------------- %
