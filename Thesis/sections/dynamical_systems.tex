% -------------------------------------------------------------------------------------- %

\subsection{Motivation}

Our goal is to investigate the qualitative, long-term behavior of systems in which a given 
function describes the trajectory of a point in an ambient space. Such dynamical systems 
are used in modelling physcial phenomena, economic forecasting, differential equations, etc. 
We wish to construct topological \emph{closed covers} of sets which describe the infinite 
dynamics of some portions of the system, as well as statistical \emph{invariant measures} 
which describe much larger sets in the space, but with less information. \\

The algorithms described in the present paper have been previously implemented in the 
statistical programming language \texttt{matlab} \cite*{oldGAIO}, but is now being fully refactored and
reimplemented in the open-source, composable language \texttt{julia} \cite*{julia}. The reason for this 
change is julia's high-level abstraction capabilities, just-in-time compilation, and 
in-built set-theoretical functions, which create short, elegant code which is nonetheless 
more performant.

\subsection{Definitions}

This section should be treated as an index of definitions, to be referred back to as 
necessary during reading. In the following, we assume $M$ is a compact or a smooth manifold 
in $\mathbb{R}^d$, endowed with a metric $d$, and the 
map $f: M \rightarrow M$ is at least $\mathcal{C}^0$. Our setting is a 
\emph{discrete, autonomous dynamical system}, that is, a system of the form:

\begin{equation}
    x_{k+1} = f(x_k), \quad k = 0,\ 1,\ 2,\ \dotsc
\end{equation}

A continuous dynamical system $\dot{x} = F(x)$ can be \emph{discretized} by, for example,
considering the \emph{Poincaré time-$t$ map} over some $d-1$ dimensional hyperplane,
or by setting one "step" of the system as integrating $F$ for a set time $t$. \\

We begin by giving a set of topological definitinos of sets we wish to approximate.

\begin{definition}[(Forward-, Backward-) Invariant]
    \cite*{algGAIO} A set $A$ is called \emph{forward-invariant} if $f(A) \subset A$, 
    \emph{backward-invariant} if $f^{-1}(A) \subset A$, and \emph{invariant} if it is
    both forward- and backward-invariant.
\end{definition}

\begin{definition}[Attracting Set]
    \cite*{subalg} An invariant set $A$ is called \emph{attracting} with \emph{fundamental neighborhood}
    $U$ if for every open set $V \supset A$ there is an $N \in \mathbb{N}$ such that the tail

    \begin{equation}
        \left\{ f^k(U)\ \vert\ k \geq N \right\}    
    \end{equation}
    
    lies entirely within $A$, ie $f^k(U) \subset A\ \forall\, k \geq N$. The attracting 
    set is also called \emph{global} if the \emph{basin of attraction} 
    
    \begin{equation}
        B(A) = \bigcap\limits_{k \geq 0} f^{-k}(U)
    \end{equation}

    is the whole of $\mathbb{R}^n$.
\end{definition}

The basin of attraction acts in some sense as the set for which all points eventually arrive
in $A$. Since the map $f$ is smooth, then the closure $\bar{A}$ is invariant too. With continuity 
it becomes clear that

\begin{equation}
    A = \bigcap\limits_{k \geq 0} f^k(U).
\end{equation}

The global attractor is maximal in the sense that it contains all backward-invariant 
sets within the system. In particular, it contains all unstable manifolds.

\begin{definition}[Stable and Unstable Manifolds]
    \cite*{dynbook} Let $\bar{x}$ be a fixed point of the diffeomorphism $f$, and $U$ a neighborhood of $x$. 
    Then the \emph{local unstable manifold} is given by

    \begin{equation}
        W^u(\bar{x}, U) = \left\{x \in U\ \vert\ \lim\limits_{k \rightarrow \infty} 
        d(f^{-k}(x),\ \bar{x}) = 0\ and \ f^{-k}(x) \in U\ \forall\, k \geq 0\right\}.
    \end{equation}

    The \emph{global unstable manifold} is given by

    \begin{equation}
        W^u(\bar{x}) = \bigcup\limits_{k \geq 0} f^k(\,W^u(\bar{x}, U)\,).
    \end{equation}

    The dual definition of the \emph{(local) stable manifold} is obtained by reversing the 
    sign of $k$ in the above equations.
\end{definition}

\begin{figure}[h]
    \ctikzfig{unstablemanifold}
    \caption{\cite*{dynskript} Stable and unstable manifolds, local and global}
    \label{fig:manifold}    
\end{figure}

\begin{definition}[Pseudoperiodic]
    \cite*{dynbook} Let $n\in \mathbb{N}$. A set 
    $\left\{ x_k\ \vert\ k \in \left\{ 1,\ \dotsc ,\ n \right\} \right\}$ 
    is called \emph{$\epsilon$-pseudoperiodic} if for any $k$

    \begin{equation}
        d(x_k,\ x_{k + 1\, \text{mod}\, n}) < \epsilon.
    \end{equation}

\end{definition}

As the name suggests, an $\epsilon$-pseudoperiodic orbit is "almost" periodic inthe sense 
that it represents a "small" perturbation of a theoretically periodic orbit. In practice,
such direct orbits may not be known, but it will preresent a naturally useful definition
in our approximations.

\begin{figure}[h]
    \ctikzfig{pseudoperiodic}
    \caption{\cite*{dynbook} A $0.1$-pseudoperiodic orbit of the map $f(x,y) = (y,\ 0.05\, (1 - x^2)\, y - x)$}
    \label{fig:pseudoperiodic}
\end{figure}

\begin{definition}[Chain Recurrent]
    \cite*{dynbook} The point $\bar{x} \in M$ is called \emph{chain recurrent} if for any $\epsilon > 0$ 
    there exists an $\epsilon$-pseudoperiodic orbit. The \emph{chain recurrent set} $R_M(f)$ 
    is the set of all chain recurrent points in $M$.
\end{definition}

As shown in \cite*{algGAIO} we have the inclusion $R_M(f) \subset \bigcap\limits_{k \geq 0} f^k(M)$,
which also shows that $R_M(f)$ is an invariant set. \\

We finish with a measure-theoretical definition for types of measures we wish to approximate.

\begin{definition}[Perron-Frobeniuns Operator, Invariant Measure]
    \cite*{attr} Let $\mathcal{M}$ be the set of probability measures on $M$. We define the
    \emph{Perron-Frobenius operator} $P:\mathcal{M} \rightarrow \mathcal{M}$ pointwise
    on the level of sets as

    \begin{equation}
        (P\mu)(A) = \mu \circ f^{-1}(A).
    \end{equation}

    A measure $\mu$ is called \emph{invariant} if it is an eigenmeasure of $P$ with
    eigenvalue $1$, ie $P\mu = \mu$.
\end{definition}

\begin{remark}
    The Perron-Frobenius operator is often also called \emph{transfer operator} or
    \emph{pushforward operator}. 
\end{remark}

An invariant measure can be used to understand the global behavior of a dynamical 
system, with more $\mu$-mass assigned to regions which are visited frequently over long
trajectories, and less $\mu$-mass to regions visited less frequently. \\

An eigenvalue of $1$ is not the only object of interest when considering the operator $P$.
Suppose instead we have a finite (complex valued) measure with $P \nu = \lambda \nu$ for a 
$\lambda = -1$. Then, using finiteness and borel measurability, we can find 
a partition of M in two disjoint subsets $A_1, A_2$ such that $\nu (A_1) = - \nu (A_2)$.
In particular, this implies that $f$ maps $A_1$ to $A_2$, and $A_2$ to $A_1$ 
(since $P^2 \nu = \nu$). This paritition forms a \emph{two-cycle}.
