\documentclass[draft]{article}
\usepackage[utf8]{inputenc}
\usepackage{amsmath}
\usepackage{amsfonts}
\usepackage{fontspec}
\setmonofont{FreeMono}
\setlength{\parindent}{0pt}

\title{EWS Blatt 10}
\author{April Herwig}

\begin{document}

\maketitle

$\mathbb{E}\left[ \left( \hat{h}(x) - \mathbb{E}\left[ \hat{h}(x) \right] \right) ^2 \right]$
$ \tilde{X} = [1\quad X], \quad\tilde{y}^T = [1\quad y^T] $
$ \theta ^{*} = \text{argmin}_\theta\  \| \tilde{y} - \tilde{X}\theta \| + \lambda\left(\| \tilde{I}\theta \|^2 \right)  $
\[
    \tilde{I}=\begin{bmatrix}
        0 & & & \\
        & 1 & & \\
        & & \ddots & \\
        & & & 1
    \end{bmatrix} = I - \begin{bmatrix}
        1 & & & \\
        & 0 & & \\
        & & \ddots & \\
        & & & 0 
    \end{bmatrix}
\]
\begin{align*}
    \hat{\theta} &= \text{argmin}_\theta\ R(\theta) \\
    &= \text{argmin}_\theta\ \| \tilde{y} - \tilde{X}\theta \|^2 + \lambda\| \tilde{I}\theta \| ^2 \\
    &= \langle \tilde{y} - \tilde{X}\theta,\ \tilde{y} - \tilde{X}\theta \rangle + \lambda\langle\tilde{I}\theta,\ \tilde{I}\theta\rangle \\
    &= \tilde{y}^T \tilde{y} +\theta^T\tilde{X}^T\tilde{X}\theta - 2\theta^T\tilde{X}^T\tilde{y} + \lambda\theta^T\tilde{I}^T\tilde{I}\theta \\
\end{align*}
$ \left(\tilde{X}^T\tilde{X} + \tilde{I}^T\tilde{I}\right)\hat{\theta} = \tilde{X}^T\tilde{y} $
$ \tilde{X}^T\tilde{X} $
$ \binom{4.2}{5.6}  \nabla f(x) = \frac{x - b}{\| x - b \|} x - \nabla f(x) $
$ \sum_{k = 0}^{\infty} (5/6)^k  $

\begin{verbatim*}
    test
\end{verbatim*}

$ Q_\epsilon^k = \{z : \frac{(2k-1)\pi}{2} + \epsilon \leq Re(z) \leq \frac{(2k-1)\pi}{2} + \epsilon, \quad |Im(z)| \leq \epsilon \} $ 
$ N_f(0, Q_\epsilon^k) - N_f(\infty, Q_\epsilon^k) = \frac{1}{2\pi i} \int_{dQ_\epsilon^k} \frac{f'(z)}{f(z)} \,dz $
$ =  \frac{1}{2\pi i} \int_{dQ_\epsilon^k} \frac{sec^2(z)-1}{tan(z)-z} \,dz = res_{z = \frac{(2k+1)\pi}{2}} \frac{sec^2(z)-1}{tan(z)-z} = -1$ 

1. Relabel s.d. $a \leq b$\newline
2. $a = 0\quad\Rightarrow\quad$  return b\newline
3. rechne $b = ka+r$, \newline
4. setzte $b = r$, repeat


\end{document}