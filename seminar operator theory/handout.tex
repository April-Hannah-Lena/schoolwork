\documentclass[a4paper]{article}
\usepackage[utf8]{inputenc}
\usepackage{amsmath}
\usepackage{amssymb}
\usepackage{amsthm}
\usepackage{amsfonts}
\usepackage[a4paper]{geometry}
\usepackage{parskip}
\usepackage{hyperref}
\usepackage{biblatex}

\newcommand{\Tr}{\textrm{Tr}\,}
\newcommand{\ran}{\textrm{Range}\,}
\newcommand{\bbN}{\mathbb{N}}
\newcommand{\bbR}{\mathbb{R}}
\newcommand{\bbC}{\mathbb{C}}

\theoremstyle{definition}
\newtheorem{definition}{Definition}[section]
\def\definitionautorefname{Definition}

\newtheorem{theorem}[definition]{Theorem}
\def\theoremautorefname{Theorem}
\newtheorem{proposition}[definition]{Proposition}
\def\propositionautorefname{Proposition}
\newtheorem{lemma}[definition]{Lemma}
\def\lemmaautorefname{Lemma}
\newtheorem{corollary}[definition]{Corollary}
\def\corollaryautorefname{Corollary}

\theoremstyle{remark}
\newtheorem*{remark}{Remark}
\def\remarkautorefname{Remark}

\theoremstyle{remark}
\newtheorem{example}[definition]{Example}
\def\exampleautorefname{Ex.}

\addbibresource{handout.bib}

\title{\vspace*{-5ex}Determinants of Trace Class Operators - Lidskii's Theorem}
\author{\vspace*{-15ex}April Herwig}
\date{11.07.2023}

\begin{document}

\maketitle

\begin{definition}[Notation]
    We consider a separable Hilbert space $X$ with inner product $(\cdot, \cdot)$. $A \in K(X)$ is a trace-class operator. 

    Let $\lambda_1,\, \lambda_2,\, \ldots$ (resp. $s_1,\, s_2,\, \ldots$) be the eigenvalues (resp. singular values) of $A$ in descending order:
    \begin{equation}
        A\, v_j = \lambda_j v_j, \quad |\lambda_1| \geq |\lambda_2| \geq \ldots, \quad s_1 \geq s_2 \geq \ldots\, .
    \end{equation}
    $A$ can be decomposed as $A = U\, |A|$ where $U$ is unitary and $|A|$ is positive and self-adjoint. $\left\{ z_j \right\}_j$ is an orthonormal basis of $\overline{\ran |A|}$. Writing $w_j = U z_j$, we have
    \begin{equation}
        A = \sum_{j=1}^\infty s_j (\cdot, z_j) w_j \,. 
    \end{equation}
\end{definition}

\subsection{Lidskii's Theorem}

\begin{theorem}[Lidksii]
    \label{lidskii}
    Let $X$ be a Hilbert space and $A \in L(X)$ be a trace-class operator with eigenvalues $\lambda_1,\, \lambda_2,\, \ldots$. Then 
    \begin{equation}
        \Tr A = \sum_{n=1}^{\infty} \lambda_n\ . 
    \end{equation}
\end{theorem}

\subsection{Tools: Generalized Eigenspaces and a Singular Value Inequality}

\begin{lemma}[Generalized eigenspaces]
    Let $0 \neq \lambda \in \sigma (A)$. Then there exist subspaces $N = N(\lambda)$, $R = R(\lambda)$ of $X$ with
    \begin{enumerate}
        \item[a)] (finiteness) $\dim N < \infty$,
        \item[b)] (splitting) $X = N \oplus R$,
        \item[c)] (invariance) $A(N) \subset N,\quad A(R) \subset R$,
        \item[d)] (isolation) $\lambda \in \sigma (A \vert N),\quad \lambda \in \sigma (A \vert R)$ . 
    \end{enumerate}
    Write $P_\lambda : X \twoheadrightarrow N$ for the orthogonal projection of $X$ into $N$. 
\end{lemma}

\begin{lemma}[Lalesco-Schur-Weyl]
    For any $N \in \bbN,\,\ \prod_{j=1}^N | \lambda_j | \leq \prod_{j=1}^N s_j$. 
    In particular we can conclude the eigenvalue - singular value inequality 
    \begin{equation}
        \sum_{j=1}^{\infty} | \lambda_j | \leq \sum_{j=1}^{\infty} s_j . 
    \end{equation}
\end{lemma}

\begin{lemma}[An infinite product formulation]
    \label{prod}
    Let $f : \bbC \circlearrowleft$ be an entire function with $f(0) = 1$. Let $\left\{ z_j \right\}_j$ be the zeros of $f$ and assume $\sum_{j=1}^{\infty} | z_j |^{-1} < \infty$. Finally suppose the boundedness condition
    \begin{equation}
        \forall\, \epsilon > 0\,\  \exists\, C > 0 \,:\,\ | f(z) | \leq C \cdot e^{\epsilon |z|}\,\ \forall\, z \in \bbC . 
    \end{equation}
    Then 
    \begin{equation}
        f(z) = \prod_{j=1}^{\infty} (1 - \frac{z}{z_j}) . 
    \end{equation}
\end{lemma}

\subsection{(Antisymmetric) Tensor Products in Hilbert Spaces}

\begin{definition}
    For $x_1, x_2, \ldots, x_n \in X$, let $x_1 \otimes x_2 \otimes \ldots \otimes x_n$ be the multilinear map 
    \begin{equation}
        (x_1 \otimes x_2 \otimes \ldots \otimes x_n) : X \times X \times \ldots \times X \to \bbC, \quad (y_1, y_2, \ldots, y_n) \mapsto \prod_{j=1}^{n} (x_j, y_j) \,. 
    \end{equation}
    The metric completion of the span of such $x_1 \otimes \ldots \otimes x_n$ (w.r.t. the natural inner product) forms a Hilbert space, denoted $X \otimes X \otimes \ldots \otimes X$. 

    If $\left\{ e_j \right\}_j$ is an orthonormal basis of $X$, then $\left\{ e_{\iota_1} \otimes e_{\iota_2} \otimes \ldots \otimes e_{\iota_n} \right\}_{\iota_1 < \iota_2 < \ldots < \iota_n}$ is an orthonormal basis of $X \otimes X \otimes \ldots \otimes X$. 

    Operators $A_1, A_2, \ldots, A_n \in L(X)$ induce a map $X \otimes \ldots \otimes X \circlearrowleft$: 
    \begin{equation}
        (A_1 \otimes A_2 \otimes \ldots \otimes A_n)(\ell) :\  (y_1, y_2, \ldots, y_n) \mapsto \ell (A_1^* y_1, A_2^* y_2, \ldots, A_n^* y_n)
    \end{equation}

    Denote the antisymmetrization of $x_1 \otimes \ldots \otimes x_n$ as 
    \begin{equation}
        x_1 \wedge x_2 \wedge \ldots \wedge x_n = \frac{1}{\sqrt{n}}  \sum_{\pi \in S_n} (-1)^\pi\, x_{\pi (1)} \otimes x_{\pi (2)} \otimes \ldots \otimes x_{\pi (n)} 
    \end{equation}
    where $S_n$ is the symmetric group with $n$ elements, $(-1)^\pi$ is the sign of a permutation. 

    Denote by $\Lambda^n (X)$ the (Hilbert) span of such $x_1 \wedge x_2 \wedge \ldots \wedge x_n$, and $\Lambda^0 (X) = \bbC$. The same statement about orthonormal bases can be made for $\Lambda^n (X)$. 
\end{definition}

\subsection{Operator Determinants}

\begin{definition}[Fredholm's determinant]
    \begin{equation}
        \det (I + z A) = \sum_{k=0}^{\infty} z^k \cdot \Tr \Lambda^k (A) . 
    \end{equation}
\end{definition}

The technique to prove Lidskii's Theorem \ref{lidskii} is to show that Lemma \ref{prod} applies. This shows that the preceeding definition is equivalent to the following: 

\begin{definition}[Groh'berg-Krein's determinant]
    \begin{equation}
        \det (I + z A) = \prod_{j=1}^{\infty} (1 + \lambda_j z) . 
    \end{equation}
\end{definition}

%\subsection{Properties of the Determinant}

%\printbibliography[title={{\normalsize References}}]

\end{document}
