\documentclass[a4paper, 10pt]{article}
\usepackage[utf8]{inputenc}
\usepackage{amsmath}
\usepackage{amssymb}
\usepackage{amsthm}
\usepackage{amsfonts}
\usepackage[a4paper, margin=1in]{geometry}
\usepackage{parskip}
\usepackage{hyperref}
\usepackage{biblatex}

\newcommand{\Tr}{\textrm{Tr}\,}
\newcommand{\ran}{\textrm{Range}\,}
\newcommand{\bbN}{\mathbb{N}}
\newcommand{\bbR}{\mathbb{R}}
\newcommand{\bbC}{\mathbb{C}}

\theoremstyle{definition}
\newtheorem{definition}{Definition}[section]
\def\definitionautorefname{Definition}

\newtheorem{theorem}[definition]{Theorem}
\def\theoremautorefname{Theorem}
\newtheorem{proposition}[definition]{Proposition}
\def\propositionautorefname{Proposition}
\newtheorem{lemma}[definition]{Lemma}
\def\lemmaautorefname{Lemma}
\newtheorem{corollary}[definition]{Corollary}
\def\corollaryautorefname{Corollary}

\theoremstyle{remark}
\newtheorem*{remark}{Remark}
\def\remarkautorefname{Remark}

\theoremstyle{remark}
\newtheorem{example}[definition]{Example}
\def\exampleautorefname{Ex.}

\addbibresource{handout.bib}

\title{\vspace*{-5ex}Determinants of Trace Class Operators - Lidskii's Theorem}
\author{\vspace*{-15ex}April Herwig}
\date{11.07.2023}

\begin{document}

\maketitle

\section{Lidskii's Theorem}

\begin{theorem}[Lidksii]
    \label{lidskii}
    Let $X$ be a Hilbert space and $A \in L(X)$ be a trace-class operator with eigenvalues $\lambda_1,\, \lambda_2,\, \ldots$. Then 
    \begin{equation}
        \Tr A = \sum_{n=1}^{\infty} \lambda_n\ . 
    \end{equation}
\end{theorem}

\section{Tools: Generalized Eigenspaces and a Product Form}

\begin{definition}[Notation]
    We consider a separable Hilbert space $X$ with inner product $(\cdot, \cdot)$. $A \in K(X)$ is a trace-class operator. 

    Let $\lambda_1,\, \lambda_2,\, \ldots$ (resp. $s_1,\, s_2,\, \ldots$) be the eigenvalues (resp. singular values) of $A$ in descending order:
    \begin{equation}
        A\, v_j = \lambda_j v_j, \quad |\lambda_1| \geq |\lambda_2| \geq \ldots, \quad s_1 \geq s_2 \geq \ldots\, .
    \end{equation}
    $A$ can be decomposed as $A = U\, |A|$ where $U$ is unitary and $|A|$ is positive and self-adjoint. $\left\{ z_j \right\}_j$ is an orthonormal basis of $\overline{\ran |A|}$. Writing $w_j = U z_j$, we have
    \begin{equation}
        A = \sum_{j=1}^\infty s_j (\cdot, z_j) w_j \,. 
    \end{equation}
\end{definition}

\begin{theorem}[Generalized eigenspaces]
    Let $0 \neq \lambda \in \sigma (A)$. Then there exist subspaces $N = N(\lambda)$, $R = R(\lambda)$ of $X$ with
    \begin{align}
        \text{a)}\ &\ \text{(finiteness)} \quad\ \dim N < \infty, \\
        \text{b)}\ &\ \text{(splitting)} \quad\ \ X = N \oplus R, \\
        \text{c)}\ &\ \text{(invariance)} \quad A(N) \subset N,\,\ A(R) \subset R, \\
        \text{d)}\ &\ \text{(isolation)} \quad\ \ \lambda \in \sigma (A \,\vert\, N),\,\ \lambda \in \sigma (A \,\vert\, R). 
    \end{align}
    Write $P_\lambda : X \twoheadrightarrow N$ for the orthogonal projection of $X$ into $N$. 
\end{theorem}

\begin{lemma}[Hadamard factorization]
    \label{prod}
    Let $f : \bbC \circlearrowleft$ be an entire function with $f(0) = 1$. Let $\left\{ z_j \right\}_j$ be the zeros of $f$ and assume $\sum_{j=1}^{\infty} | z_j |^{-1} < \infty$. Finally suppose the boundedness condition
    \begin{equation}
        \forall\, \epsilon > 0\,\  \exists\, C > 0 \,:\,\ | f(z) | \leq C \cdot e^{\epsilon |z|}\,\ \forall\, z \in \bbC . 
    \end{equation}
    Then 
    \begin{equation}
        f(z) = \prod_{j=1}^{\infty} (1 - \frac{z}{z_j}) . 
    \end{equation}
\end{lemma}

\begin{lemma}[Weyl-Horn]
    We have the singular value inequality 
    \begin{equation}
        \sum_{j=1}^{\infty} | \lambda_j | \leq \sum_{j=1}^{\infty} s_j . 
    \end{equation}
\end{lemma}

\section{(Antisymmetric) Tensor Products in Hilbert Spaces}

\begin{definition}
    For $x_1, x_2, \ldots, x_n \in X$, let $x_1 \otimes x_2 \otimes \ldots \otimes x_n$ be the multilinear map 
    \begin{equation}
        (x_1 \otimes x_2 \otimes \ldots \otimes x_n) : X \times X \times \ldots \times X \to \bbC, \quad (y_1, y_2, \ldots, y_n) \mapsto \prod_{j=1}^{n} (x_j,\, y_j) \,. 
    \end{equation}
    The (Hilbert) span of such $x_1 \otimes \ldots \otimes x_n$ (w.r.t. the natural inner product) 

    \begin{equation}
        \left\langle\, x_1 \otimes x_2 \otimes \ldots \otimes x_n,\,\ y_1 \otimes y_2 \otimes \ldots \otimes y_n \,\right\rangle = \prod_{j=1}^{n} (x_j,\, y_j) \,. 
    \end{equation}
    
    forms a Hilbert space, denoted $X \otimes X \otimes \ldots \otimes X$. 

    If $\left\{ e_j \right\}_j$ is an orthonormal basis of $X$, then $\left\{ e_{\iota_1} \otimes e_{\iota_2} \otimes \ldots \otimes e_{\iota_n} \right\}_{\iota_1 < \iota_2 < \ldots < \iota_n}$ is an orthonormal basis of $X \otimes X \otimes \ldots \otimes X$. 

    Operators $A_1, A_2, \ldots, A_n \in L(X)$ induce a map $X \otimes \ldots \otimes X \circlearrowleft$: 
    \begin{equation}
        (A_1 \otimes A_2 \otimes \ldots \otimes A_n)(\ell) :\  (y_1, y_2, \ldots, y_n) \mapsto \ell (A_1^* y_1,\, A_2^* y_2,\, \ldots,\, A_n^* y_n)
    \end{equation}

    Denote the antisymmetrization of $x_1 \otimes \ldots \otimes x_n$ as 
    \begin{equation}
        x_1 \wedge x_2 \wedge \ldots \wedge x_n = \frac{1}{\sqrt{n}}  \sum_{\pi \in S_n} (-1)^\pi\, x_{\pi (1)} \otimes x_{\pi (2)} \otimes \ldots \otimes x_{\pi (n)} 
    \end{equation}
    where $S_n$ is the symmetric group with $n$ elements, $(-1)^\pi$ is the sign of a permutation. 

    Denote by $\Lambda^n (X)$ the (Hilbert) span of such $x_1 \wedge x_2 \wedge \ldots \wedge x_n$. $\Lambda^0 (X) = \bbC$, $\Lambda^1 (X) = X$. The same statement about orthonormal bases can be made for $\Lambda^n (X)$. 
\end{definition}

\begin{proposition}[Properties of the tensor products] 
    \begin{align}
        \text{a)} &\quad (A_1 \otimes A_2 \otimes \ldots \otimes A_n) (x_1 \otimes x_2 \otimes \ldots \otimes x_n) = (A_1 x_1) \otimes (A_2 x_2) \otimes \ldots \otimes (A_n x_n), \\
        \text{b)} &\quad | \Lambda^n (A) | = \Lambda^n (|A|), \\
        \text{c)} &\quad \Lambda^n (A)\ \text{has singular values}\ s_{\iota_1} \cdot s_{\iota_2} \cdot \ldots \cdot s_{\iota_n}\ \forall\, \iota_1 < \iota_2 < \ldots < \iota_n, \\
        \text{d)} &\quad \text{If}\ \dim X = d < \infty,\ \text{then}\ \Tr \Lambda^d (A) = \det A,\,\ \Tr \Lambda^{d+m} (A) = 0\,\ \forall\, m > 0. 
    \end{align}

\end{proposition}

\section{Operator Determinants}

\begin{definition}[Fredholm's determinant]
    \begin{equation}
        \det (I + z A) = \sum_{k=0}^{\infty} z^k \cdot \Tr \Lambda^k (A) . 
    \end{equation}
\end{definition}

The technique to prove Lidskii's Theorem \ref{lidskii} is to show that Lemma \ref{prod} applies. This shows that the preceeding definition is equivalent to the following: 

\begin{definition}[Groh'berg-Krein's determinant]
    \begin{equation}
        \det (I + z A) = \prod_{j=1}^{\infty} (1 + \lambda_j z) . 
    \end{equation}
\end{definition}

%\section{Properties of the Determinant}

%\printbibliography[title={{\normalsize References}}]

\end{document}
