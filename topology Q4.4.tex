%%%%%%%%%%%%%%%%%%%%%%%%%%%%%%%%%%%%%%%%%%%%%%%%%%%%%%%%%%%%%%%%%%%%%%%%%%%%%%%%
% Please keep your source code to 80 characters per line.                      %
                                                                               %
\documentclass{article}                                                        %
                                                                               %
                                                                               %
% basic packages:                                                              %
\usepackage{amsmath, amssymb, mathtools}                                       %
\usepackage{hyperref}                                                          %
\usepackage{enumitem}                                                          %
                                                                               %
                                                                               %
%% Here you can add additional packages.                                       %
% Please add comments explaining why those packages are necessary.             %
% For example:
% \usepackage{tikz-cd}% to typeset commutative diagrams

%% End of additional packages                                                  %
                                                                               %
                                                                               %
                                                                               %
% produces a dummy solution environment,                                       %
% which will have more functionality in the assembled solution document.       %
\newtheorem{loesung}{Solution}                                                 %
                                                                               %
                                                                               %
\begin{document}                                                               %
                                                                               %
\begin{loesung}                                                                %
  [April Herwig]% please fill in your name
  {4.4}% fill in sheet-number & exercise-number (e.g. 3.14).
                                                                               %
  %% Start writing your solution here.                                         %

  \textit{a)} Let $Y = \prod\limits_{n\in\mathbb{N}} X_i$. For each 
  $n\in\mathbb{N}$ define $\delta_n (x,y) = \rho \circ d_{X_n} (x,y)$ where
  $d_{X_n}$ is the metric on $X_n$ and $\rho$ is a monotone increasing 
  homeomorphism $\rho : [0,\infty) \rightarrow [0,1)$, eg. 
  $x \mapsto \frac{x}{1+x}$. Define the metric on $Y$ as 

  \begin{equation*}
    d : Y \times Y \rightarrow [0,\infty),\ 
    (x,y) \mapsto \sum\limits_{n\in\mathbb{N}}\frac{1}{2^n}\,\delta_n (x,y)\,.
  \end{equation*}

  This clearly satisfies the conditions for a metric since each $d_{X_n}$ is a 
  metric and $\rho$ is monotone increasing. We will show that this metric
  induces the same topology as the product topology.

  Let $U \subseteq Y$ be open in the product topology, ie. 
  \begin{equation*}
    U = \prod\limits_{n\in\mathbb{N}} U_n,\quad 
    U_n\ \text{open},\quad 
    \exists\, N\in\mathbb{N}\,:\, U_m = X_m\ \forall m > N\,.
  \end{equation*}

  For $x\in U$, each space $X_n,\ n\in\mathbb{N}$ has an $\epsilon_n > 0$ with
  $B_{\epsilon_n} (x_n) \subseteq U_n$. Define 

  \begin{equation*}
    \epsilon = \text{min}
    \left\{
      \frac{1}{2^n}\rho(\epsilon_n) \vert n \leq N
    \right\}\,.
  \end{equation*}
  
  Then $\epsilon$ satisfies $x \in B_\epsilon (x) \subseteq U$ since for 
  $y \neq x$ we have

  \begin{equation*}
    d_{X_n} (y_n,x_n) > \epsilon_n \ \Leftrightarrow \ 
    \frac{1}{2^n}\delta_n (y_n,x_n) > \frac{1}{2^n}\rho(\epsilon_n)\,.
  \end{equation*}

  Hence $U$ is open in the metric topology.

  Now let $x \in Y,\ U = B_\epsilon (x) \subseteq Y,\ y \in U$. Set
  $N = \lceil \log(\frac{4}{2+\epsilon}) \rceil $ and consider

  \begin{equation*}
    V = \prod\limits_{n \leq N} B_{\epsilon / 2} (y_n)\ 
    \times\ 
    \prod\limits_{m > N} X_m\,.  
  \end{equation*}
  
  $V$ is open in the product topology and $y \in V$. Finally, we will show 
  $V \subseteq U$ and hence $U$ is a neighborhood of all its points, so open.

  \begin{gather*}
    \underbrace{
      \sum\limits_{n=1}^{N} \frac{1}{2^n}\frac{\epsilon}{2}
    }_{
      \leq \frac{\epsilon}{2}
    }
    \ + \underbrace{
      \sum\limits_{m=N+1}^{\infty} \frac{1}{2^n}
    }_{
      = 1 - 2(1 - 2^{-N})
    }
    \overset{!}{<} \epsilon \\
    1 - 2(1 - 2^{-N}) < \frac{\epsilon}{2}
    \quad\Leftrightarrow\quad
    N > \frac{\log(\frac{4}{2+\epsilon})}{\log(2)}\,.
  \end{gather*}

  \textit{b)} Let $X = \left\{ 0,1 \right\}$ equipped with the discrete
  topology. $X$ is trivially metrizable. Consider 

  \begin{equation*}
    Y = X^{\mathbb{R}} 
    = \prod\limits_{x\in\mathbb{R}} \left\{ 0,1 \right\}
    = \left\{ f : \mathbb{R} \rightarrow X \vert  \right\}\,.
  \end{equation*}
  
  Let $f_0 \equiv 0$ the zero function. A neighborhood basis of $f_0$ is given
  by

  \begin{equation*}
    \mathcal{B}_{f_0} = 
    \bigcup\limits_{\substack{K \subseteq \mathbb{R} \\ \text{finite}}}
    \underbrace{
      \left\{ f \in Y 
      \ \vert\ f(x) = 0\ \forall x \in \mathbb{R}
      \setminus K \right\}
    }_{
      =:\ U_K
    }\,.
  \end{equation*}

  Assume $Y$ is metrizable. Then it is homeomorphic to a metric space, and since
  metric spaces are first countable, Y is first countable. It follows that there
  exists a countable subset  
  $\left\{ U_{K_i} \right\}_{i\in\mathbb{N}} \subseteq  \mathcal{B}_{f_0}$
  which is still a neighborhood basis of $f_0$. In particular, 
  $\bigcup\limits_{i\in\mathbb{N}} K_i$ is countable. 
  $$\Rightarrow\ \exists\, \bar{x} \in \mathbb{R} \setminus 
  \bigcup\limits_{i\in\mathbb{N}} K_i \neq \emptyset\,.$$ Finally, consider 

  \begin{gather*}
    U_{\left\{ \bar{x} \right\}} = 
    \left\{ f \in Y\ \vert\ f(\bar{x}) = 0 \right\}
    \in \mathcal{N}_{f_0}\,, \\
    \bigcap\limits_{i\in\mathbb{N}} U_{K_i} = 
    \left\{ f \in Y\ \vert\ f(x) = 0\ 
    \forall x \in \bigcup\limits_{i\in\mathbb{N}} K_i \right\}\,.
  \end{gather*}
  
  Then $U_{\left\{ \bar{x} \right\}}$ is an open neighborhood of of $f_0$, but
  $$\bigcap\limits_{i\in\mathbb{N}} U_{K_i} 
  \not\subseteq U_{\left\{ \bar{x} \right\}}$$ which implies no $U_{K_i}$ is 
  a subset of $U_{\left\{ \bar{x} \right\}}$, a contradiction to 
  $\left\{ U_{K_i} \right\}_{i\in\mathbb{N}}$ being a neighborhood basis.

  %% End of your solution.                                                     %
                                                                               %
\end{loesung}                                                                  %
                                                                               %
\end{document}                                                                 %
%%%%%%%%%%%%%%%%%%%%%%%%%%%%%%%%%%%%%%%%%%%%%%%%%%%%%%%%%%%%%%%%%%%%%%%%%%%%%%%%