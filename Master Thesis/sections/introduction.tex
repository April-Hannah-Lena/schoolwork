% -------------------------------------------------------------------------------------- %

\section{Introduction}

Dynamical systems theory has seen many revolutions. One recent such revolution is to use 
operator theoretic tools to analyze the global behavior of chaotic systems. Concieved 
originally in the 1930's by Koopman and Von Neumann \cite{KoopmanOG} but popularized 
largely in the last two decades, the study of global dynamics under a mapping $S$ is 
conducted by studying the action of the composition $\psi \circ S$ of $S$ with an 
observable $\psi$. Observables which act as eigenvalues for the composition operator and 
its dual hold information on slowly decaying structures in phase space. 

Currently the most popular method by far for analyzing composition operators (named Koopman 
operators) is dynamic mode decomposition (DMD). The past 2 decades have seen massive growth 
in this topic, and DMD has recieved many evolutions \cite{DMDmultiverse}. One primary 
concern of nearly all DMD methods is \emph{spectral pollution}. The Koopman operator often 
has unfavorable spectral qualities such as continuous spectrum which are unstable, and are 
destroyed by finite approximation. The task set forth in the present paper is to identify 
which candidate eigenvectors (computed e.g. by DMD) are approximations of eigenmodes of 
the true Koopman operator (or its adjoint), and which arise due to discretization. 

% -------------------------------------------------------------------------------------- %