% -------------------------------------------------------------------------------------- %

\section{Background}

We consider a discrete dynamical system generated by a map $S : \Omega \to \Omega$ defined 
on a measure space $( \Omega, \scrA, dx )$ which is \emph{nonsingular}, that is 
$\int_{S^{-1} (A)} dx = 0$ for any $A \in \scrA$ with $\int_A dx = 0$. 


\subsection{Koopman and Perron-Frobenius Operators}

Koopman operator theory shifts the focus from dynamics of points in state space $\scrX$ to 
dynamics of observables $g : \Omega \to \bbC$. A state $x \in \Omega$ evolves by iteratively 
applying the map $S$, and similarly observables evolve under the action of the 
\emph{Koopman operator}

\begin{equation}
    \scrK g = g \circ S . 
\end{equation}

The primary benefit of this alternative viewpoint of dynamics is that $\scrK$ is 
\emph{linear} and can therefore be analyzed using algebraic and functional analytic tools. 
However, the Koopman operator acts on function spaces which are typically 
infinite-dimensional. Even the Krylov space 
$\left\{ g,\, \scrK g,\, \scrK^2 g,\, \ldots \right\}$ is generically infinite-dimensional: 
consider e.g. an indicator function $g = \bbone_{\left[ 0, 1 \right]}$ and a translation 
$S (x) = x - 1$. 

It should be noted that until now we have not declared a function space to act as a domain 
for $\scrK$. This is because there are many potential domains for which $\scrK$ is not 
only well-defined, but has interesting properties worth studying. For the current section 
we consider $\scrK : L^\infty \to L^\infty$. On this space $\scrK$ is obviously bounded, 
in fact it is a contraction. 

The Koopman operator tracks how observables evolve with $S$. There is however a dual 
viewpoint: that of \emph{densities} and \emph{pushforwards}. From the duality pairing

\begin{equation}
    \label{eq:duality}
    \left\langle f,\ \scrK g \right\rangle = \left\langle \scrL f,\ g \right\rangle
\end{equation}

for $f \in L^1,\ g \in L^\infty$ we may deduce the form of an adjoint operator, known as 
the \emph{transfer operator} or \emph{Perron-Frobenius operator}. Consider an indicator 
$g = \bbone_A$ for an $A \in \scrA$ and let $f \geq 0$. The left hand side of 
\ref{eq:duality} is

\begin{equation}
    \int f (x) \, \bbone_A ( S (x) )\ dx 
    = \int f (x) \, \bbone_{S^{-1} (A)} (x)\ dx 
    = \int_{S^{-1} (A)} f (x)\ dx , 
\end{equation}

and the right hand side 

\begin{equation}
    \int \scrL f (x)\, \bbone_A (x)\ dx
    = \int_A \scrL f (x)\ dx . 
\end{equation}

Hence $\scrL f \in L^1$ should satisfy the equation 

\begin{equation}
    \label{eq:defLf}
    \int_A \scrL f (x)\ dx = \int_{S^{-1} (A)} f (x)\ dx . 
\end{equation}

A short exercise in measure theory shows that 

\begin{equation}
    \label{eq:pushforward}
    A \mapsto \int_{S^{-1} (A)} f (x)\ dx
\end{equation}

defines a finite absolutely continuous measure, and hence by the Radon-Nikodym theorem 
the measure has a unique density which by \ref{eq:defLf} is precisely $\scrL f$. For 
general $f \in L^1$ decompose $f = f^+ - f^-$ with $f^+, f^- \geq 0$ and set 
$\scrL f = \scrL f^+ - \scrL f^-$. 

\ref{eq:pushforward} shows that $\scrL$ is precisely the action of the pushforward 
$S_\sharp \mu = \mu \circ S^{-1}$ for densities. This provides a useful intuition for 
the transfer operator: we move away from the viewpoint of single points $x \in \scrX$, 
and instead consider \emph{distributions} of point density. If $x$ is 
chosen from a random distribution with density $f$, then the distribution of $S (x)$ has 
density $\scrL f$. 


\subsection{Spectral Properties}

\subsubsection{The Spectrum}

We begin with abstract definitions for parts of the spectrum. 

\begin{definition}
    Let $M : D(M) \to \scrX$ be a linear operator on a Banach space 
    $(X, \| \cdot \|)$. The domain $D(M) \subset \scrX$ of $M$ 
    may be a dense subset of $\scrX$. $M$ is called \emph{closed} if: whenever 
    $x_k \to x$ is a convergent sequence with $M x_k$ convergent, then $M x_k \to M x$. 
\end{definition}

Closedness is weaker than continuity due to the extra assumption that $M x_k$ is 
convergent. 

\begin{definition}
    The \emph{spectrum} $\sigma (M)$ of a closed operator $M$ is the set of numbers 
    $\lambda \in \bbC$ for which $M - \lambda I$ does not have a bounded inverse. The 
    complement $\rho (M) \defeq \bbC \setminus \sigma (M)$ is called \emph{resolvent set} 
    and $( M - \lambda I )^{-1}$ is called the \emph{resolvent}. 
\end{definition}

\begin{proposition}
    The spectrum admits a decomposition 
    $\sigma (M) = \sigma_p (M) \oplus \sigma_c (M) \oplus \sigma_r (M)$ into \emph{point-, 
    continuous-,} and \emph{residual-spectrum} with
    \begin{align}
        &\sigma_p (M) = \left\{ \lambda \in \bbC \mid 
            M - \lambda I \text{ is not injective} \right\}, &\phantom{.} \\
        &\sigma_c (M) = \left\{ \lambda \in \bbC \mid 
            M - \lambda I \text{ is injective, but its range is a dense subset of } 
            \scrX \right\}, \\
        &\sigma_r (M) = \left\{ \lambda \in \bbC \mid 
            M - \lambda I \text{ is injective, but does not have dense range} \right\}. 
    \end{align}
\end{proposition}

\begin{remark}
    In some literature, $0$ is never part of the spectrum. We do not take this approach. 
\end{remark}

An element $\lambda \in \sigma_p (M)$ is a \emph{true} eigenvalue in the sense that there 
exists a vector $x \in \scrX$ such that $(M - \lambda I) x = 0$. The continuous spectrum 
has a similar characterization as the set of $\lambda \in \bbC$ for which $M _ \lambda I$ 
is injective but \emph{not} bounded from below, i.e. there exists a sequence $(x_k)_k$, 
$\| x_k \| = 1$, for which $(M - \lambda I) x_k \to 0$. Equivalently, 
$(M - \lambda I)^{-1}$ cannot be extended to a bounded linear operator, but is still a 
(densely defined) closed linear operator. 

One might ask why such detail is required in defining the spectrum. The answer lies in 
the subtle consequences of infinite-dimensionality. The following example from 
\cite{spectraexample} shows the subtle interplay between residual spectrum and spectrum 
of the adjoint operator, all of which requires infinite-dimensionality. 

\begin{example}[Shift operators on $\ell^2 (\bbN)$]
    \label{ex:shift}
    The canonical right-shift operator

    \begin{equation}
        R : (a_1, a_2, \ldots) \mapsto (0, a_1, a_2, \ldots)
    \end{equation}

    is adjoint to the left shift

    \begin{equation}
        L : (a_1, a_2, \ldots) \mapsto (a_2, a_3, \ldots) . 
    \end{equation}

    We claim: 

    \begin{align}
        \label{eq:pointspec}
        &\sigma_p (L) = \left\{ \lambda \in \bbC \mid | \lambda | < 1 \right\} &
        &\sigma_p (R) = \emptyset \\
        \label{eq:contspec}
        &\sigma_c (L) = \left\{ \lambda \in \bbC \mid | \lambda | = 1 \right\} &
        &\sigma_c (R) = \left\{ \lambda \in \bbC \mid | \lambda | = 1 \right\} \\
        \label{eq:resspec}
        &\sigma_r (L) = \emptyset & 
        &\sigma_r (R) = \left\{ \lambda \in \bbC \mid | \lambda | < 1 \right\} . 
    \end{align}

    Indeed, to see \ref{eq:pointspec} consider $| \lambda | < 1$, 
    $a^\lambda = (\lambda, \lambda^2, \ldots)$. Then clearly 
    $L a^\lambda = \lambda a^\lambda$. On the 
    other hand, suppose $R a = \lambda a$ for some $a \neq 0$, $\lambda \in \bbC$. Then 
    letting $n$ be the first index such that $a_n \neq 0$, we must have 
    $0 = a_{n-1} = (R a)_n = \lambda a_n$ so $\lambda = 0$. But the kernel of $R$ is 
    clearly $\left\{ 0 \right\}$. 

    To see \ref{eq:contspec}, take $| \lambda | = 1$ and construct \emph{approximate} 
    $(L, \lambda)$-eigenvectors 
    $v^n = \left( \lambda, \lambda^2, \ldots, \lambda^n, 0, \ldots \right)$. Then 
    $(L - \lambda I) v^n = (0, \ldots, 0, \lambda^{n+1}, 0, \ldots)$ so 
    $\| (L - \lambda I) v^n \| = 1$ but $\| v^n \| = \sqrt{n}$. It remains to show 
    that $\lambda$ is of continuous spectrum, as opposed to residual spectrum. Indeed, 
    for any $w$ such that for all $v \in \ell^2$, 
    $0 = \left\langle (L - \lambda I) v, w \right\rangle 
    = \left\langle v, (R - \bar{\lambda} I) w \right\rangle$ so that $w$ is an eigenvector of 
    $R$ which by \ref{eq:pointspec} is a contradiction. An entirely analogous argument 
    shows that the circle is also continuous spectrum for $R$. 

    We now reverse the discussion to see that any $| \lambda | < 1$ is in the 
    residual spectrum of $R$. For $v \in \ell^2$ and $w$ a $\bar{\lambda}$-eigenvector of 
    $L$ we have 
    $0 = \left\langle v, 0 \right\rangle 
    = \left\langle v, (L - \bar{\lambda} I) w \right\rangle 
    = \left\langle (R - \lambda I) v, w \right\rangle$ so that the range of 
    $R - \lambda I$ is orthogonal to $w$. 

    Finally, $\sigma (M)$ is bounded by $\| M \|$ for any operator $M$, and clearly 
    $\| L \| = 1$. But $\sigma_p (L) \cup \sigma_c (L) = 
    \left\{ \lambda \in \bbC \mid | \lambda | \leq 1 \right\}$ so 
    $\sigma_r (L) = \emptyset$. Combined with the previous paragraph, this shows 
    \ref{eq:resspec}. 

\end{example}

The example already suggests some important relationships regarding the spectrum of 
adjoint operators. 

\begin{theorem}
    \label{thm:spec_relations_adjoints}
    Let $M$ be a densely defined closed linear operator on a Hilbert space. 
    
    \begin{enumerate}
        \item $\lambda \in \sigma (M)$ iff $\bar{\lambda} \in \sigma (M^*)$. 
        \item If $\lambda \in \sigma_r (M)$, then $\bar{\lambda} \in \sigma_p (M^*)$. 
        \item Conversely, if $\lambda \in \sigma_p (M)$, then 
              $\bar{\lambda} \in \sigma_p (M^*) \cup \sigma_r (M^*)$. 
        \item $\sigma_c (M) = \sigma_c (M^*)$. 
    \end{enumerate}
\end{theorem}

\begin{proof}
    \begin{enumerate}
        \item Suppose $M - \lambda I$ has a bounded inverse $B$. Then 
              $(M - \lambda I) B = B (M - \lambda I) = I$. Equivalently 
              $B^* (M^* - \bar{\lambda} I) = (M^* - \bar{\lambda} I) B^* = I^* = I$. 
        \item The range of $M - \lambda I$ is not dense in $\scrX$. Hence there exists a 
              $v \perp \text{ran} (M - \lambda I)$. But this implies 
              $v \in \text{ker} (M^* - \bar{\lambda} I)$. 
        \item There exists a $v \in \text{ker} (M - \lambda I)$ which implies 
              $v \perp \text{ran} (M^* - \bar{\lambda} I)$. 
        \item This follows from $1$, $2$, and $3$. 
    \end{enumerate}
\end{proof}


\subsubsection{Spectra of Koopman and Perron-Frobenius Operators}

The Koopman and Perron-Frobenius operator spectrum holds information about the long-term 
mixing rates of structures in phase space. We write (with some abuse of notation) 
$\left. \scrL \right|_{\scrX}$ and $\left. \scrK \right|_{\scrX}$ to denote the 
Perron-Frobenius / Koopman operators acting on the domain $\scrX$. 

\begin{definition}
    An eigenfunction $f$ for $\left. \scrL \right|_{L^1}$ with eigenvalue $1$ is the 
    density of a \emph{(signed) invariant measure}. We say $S$ \emph{preserves the 
    measure $f\,dx$}. 
\end{definition}

The following are well-known facts about the Koopman and Perron-Frobenius operator. They 
can be found in many sources e.g. \cite{lasotamackey}. 

\begin{theorem}
    \begin{enumerate}
        \item If $S$ is ergodic\footnote{Ergodicity and mixing describe how obeservations 
            become decorrelated over time. $S$ is \emph{ergodic} if for all $A, B \in \scrA$ 
            we have $\frac{1}{n}\sum_{j=0}^{n-1} \int_{ S^{-j} (A) \cap B } dx \to 
            \int_A dx \int_B dx$ as $n \to \infty$. $S$ is weak-mixing if we have 
            $\frac{1}{n}\sum_{j=0}^{n-1} \left| \int_{ S^{-j} (A) \cap B } dx 
            - \int_A dx \int_B dx \right| \to 0$ as $n \to \infty$. },
            then there is at most one invariant density. Conversely, if there is a unique 
            invariant density which is $(dx-)$almost everywhere positive, then $S$ is 
            ergodic. 
        \item If $S$ is ergodic, then every eigenvalue of $\left. \scrK \right|_{L^1}$ 
            is simple. 
        \item Suppose $S$ is invertible. Then $S$ is weak-mixing iff $1$ is the only 
            eigenvalue of $\left. \scrK \right|_{L^1}$. 
    \end{enumerate}
\end{theorem}

The following theorem is from \cite{attr}. 

\begin{definition}
    A set $A \subset \Omega$ in phase space is called \emph{$\delta$-almost-invariant} if 
    $\int_{ S^{-1} (A) \cap A } dx = \delta \cdot \int_A dx$. 
\end{definition}

\begin{theorem}
    Let $\bbR \ni \lambda < 1$ be an eigenvalue corresponding to a real-valued 
    normalized eigenfunction $f$ of $\left. \scrL \right|_{L^1}$. Let further 
    $A \subset \Omega$ be such that $\int_A f\, dx = \frac{1}{2}$. Then 

    \begin{equation}
        \delta + \eta = \lambda + 1
    \end{equation}

    if $A$ is $\delta$-almost-invariant and $\Omega \setminus A$ is 
    $\eta$-almost-invariant. 
\end{theorem}

These are only a few interesting properties of the spectrum of these operators, but there 
are many more which can be studied. 


\subsection{Pseudospectra}

The reder will have likely noticed the sensitivity required in understanding the spectrum 
for infinite-dimensional operators. In particular the spectral types can be unstable 
w.r.t. perturbations of the operator. For example, for any self-adjoint operator $M$ (e.g. 
the generator for the Koopman semigroup in a continuous-time dynamical system) there 
exists a compact operator $E$ with arbitrarily small norm such that the perturbation 
$M + E$ has purely point spectrum. 

The situation is even worse when one considers \emph{perturbations of the dynamics} 
instead of perturbations of the Koopman / Perron-Frobenius operators. Consider a circle 
rotation $S : \bbT \to \bbT,\ e^{2\pi i \theta} \mapsto e^{2\pi i (\theta + \alpha)}$. 
We have 
$\sigma ( \left. \scrL \right|_{L^2} ) = \sigma_p ( \left. \scrL \right|_{L^2} ) = \alpha^{\bbN_0}$. 
If the rotation is rational then the spectrum is discrete, but if the rotation is 
irrational then the spectrum is dense in the unit circle. 

In the present paper we tackle the instability problem for operator perturbations. 


\subsubsection{Definitions of the Pseudospectrum}

\begin{definition}
    Let $M : D(M) \to \scrX$ be a closed linear operator. The 
    \emph{$\epsilon$-pseudospectrum} of $M$ is 
    the smallest set in $\bbC$ which contains the spectrum of all perturbations of $M$ 
    with norm less than $\epsilon$:\footnote{
        Some authors will deine the pseudospectrum with "$\leq$" instead of "$<$". This 
        makes $\sigma_\epsilon (M)$ a closed set, but breaks theorem \ref{thm:pseudo}.}
    \begin{equation}
        \sigma_\epsilon (M) = \bigcup_{\| E \| < \epsilon} \sigma (M + E). 
    \end{equation}
\end{definition}

\begin{theorem}
    \label{thm:pseudo}
    
    We have the following equivalent formulation of the pseudospectrum:
    \begin{equation}
        \sigma_\epsilon (M) = \left\{ \lambda \in \bbC \mid 
        \| (M - \lambda I)^{-1} \| > \frac{1}{\epsilon} \right\} 
    \end{equation}
    where we use the convention that $\| (M - \lambda I)^{-1} \| = \infty$ if 
    $M - \lambda I$ is not invertible. 
\end{theorem}

\begin{proof}
    We prove "$\subset$" by contraposition: assume 
    $\| (M - \lambda I)^{-1} \| > \frac{1}{\epsilon}$ and let $\| E \| < \epsilon$. Then 
    $\| (M - \lambda I)^{-1} M \| < 1$ and hence $I + (M - \lambda I)^{-1} E$ is 
    invertible. This implies $M - \lambda I + E 
    = \left( M - \lambda I \right) \left( I + (M - \lambda I)^{-1} E \right)$ is 
    invertible. 

    Conversely, we prove "$\supset$" by showing there exists an operator $E$ with 
    $\| E \| < \epsilon$ sich that $M - \lambda I + E$ is not invertible. Since 
    $\| (M - \lambda I)^{-1} \| > \frac{1}{\epsilon}$ there exists a $u \in \scrX$ with 
    $\| u \| = 1$ and $(M - \lambda I)^{-1} u = v \in D(M)$ with 
    $\| v \| = \frac{1}{\delta} > \frac{1}{\epsilon}$.\footnote{
        At this point we require the strict inequality. Otherwise, the existence of such 
        a pair $u$, $v$ is not guaranteed. 
    } The Hahn-Banach theorem provides 
    a $v^* \in \scrX^*$ with $\| v^* \| = 1$ and $v^* v = \| v \| = \frac{1}{\delta}$. 
    Set $E = - \delta u v^*$. Then $\| E \| = \delta < \epsilon$ and 
    \begin{equation}
        E v = - \delta u v^* v = - u = - (M - \lambda I) (M - \lambda I)^{-1} u 
        = (M - \lambda I) v . 
    \end{equation}
    Rearranging provides $(M + E - \lambda I) v = 0$. 
\end{proof}

The following formulation of the pseudospectrum in the Hilbert space case will be the 
main tool we use in the section on numerical methods. 

\begin{definition}
    The function 
    \begin{equation}
        \sigma_{\inf} (M) = \inf_{\| x \| = 1} \| M x \|
    \end{equation}
    is known as the \emph{injection modulus}. 
\end{definition}

\begin{lemma}
    Let $M : D(M) \to \scrH$ be a closed linear operator on a Hilbert space. 
    If $\lambda \in \rho (M)$ then 
    \begin{equation}
        \label{eq:injection_modulus_adjoint}
        \sigma_{\inf} (M - \lambda I) = \sigma_{\inf} (M^* - \bar{\lambda} I) , 
    \end{equation}
    but this is \emph{not} necessarily true if $\lambda \in \sigma (M)$. 
\end{lemma}

\begin{proof}
    \ref{eq:injection_modulus_adjoint} for $\lambda \in \rho (M)$ follows from the fact 
    that for bounded operators $A$ we have $\| A \| = \| A^* \|$, applied to 
    $A = (M - \lambda I)^{-1}$. To see that \ref{eq:injection_modulus_adjoint} does not 
    hold for $\lambda \in \sigma (M)$, consider again the right shift from example 
    \ref{ex:shift}. Clearly $\| R x \| = \| x \|$ so $R$ is bounded from below but $0$ 
    is part of the spectrum. 
\end{proof}

\begin{theorem}
    For a Hilbert space operator $M$, 
    \begin{equation}
        \frac{1}{\left\| (M - \lambda I)^{-1} \right\|} = \min \left\{ 
            \sigma_{\inf} (M - \lambda I),\ \sigma_{\inf} (M^* - \bar{\lambda} I)
         \right\} . 
    \end{equation}
    where we use the convention that $1 / \left\| (M - \lambda I)^{-1} \right\| = 0$ 
    when $\lambda \in \sigma (M)$. 
\end{theorem}

\begin{proof}
    When $\lambda \in \sigma_p (M)$, then obviously $\sigma_{\inf} (M - \lambda I) = 0$. 
    Due to theorem \ref{thm:spec_relations_adjoints} part $2$, $\lambda \in \sigma_r (M)$ 
    has $\sigma_{\inf} (M^* - \bar{\lambda} I) = 0$. Finally, let 
    $\lambda \in \sigma_c (M)$. Assume $\sigma_{\inf} (M - \lambda I) > 0$. Then 
    $(M - \lambda I)^{-1} : \text{ran} (M - \lambda I) \to \scrW \subset \scrX$ is bounded. 
    But since $\text{ran} (M - \lambda I) \subset \scrX$ is dense, $(M - \lambda I)^{-1}$ 
    has a unique extension to $\scrX$, a contradiction. 
\end{proof}

\begin{corollary}
    In the Hilbert space setting, the psudospectrum can be formulated as: 
    \begin{align}
        \begin{split}\label{eq:pseudospectrum}
            \sigma_\epsilon (M) &= \left\{ \lambda \in \bbC \mid \exists\; u : 
            \| u \| = 1 \text{ and either }\right. \\
            &\quad\quad\quad\quad\quad\quad
            \left.\left\| (M - \lambda I) u \right\| < \epsilon \text{ or } 
            \left\| (M^* - \bar{\lambda} I) u \right\| < \epsilon \right\}    
        \end{split}
        \\[2ex]
        &= \sigma (M) \cup \left\{ \lambda \in \bbC \mid \exists\; u : 
        \| u \| = 1,\ \left\| (M - \lambda I) u \right\| < \epsilon \right\} .  
    \end{align}
\end{corollary}


\subsubsection{Properties}

% -------------------------------------------------------------------------------------- %
