% -------------------------------------------------------------------------------------- %

\section{Conclusion}

$5$ related algorithms for the study of Koopman and Perron-Frobenius operators are studied. 
The algorithms are based on the concept of \emph{pseudospectra}, which are investigated 
from a theoretical basis before applied to Koopman operators. In the process, two new 
proofs (one extension and one correction of previous proofs) are given for well-known 
results in pseudospectral theory. The idea behind each algorithm is deduced rigorously, 
and new connections between algorithms are presented. Finally, the algorithms are explored 
on three examples. 

The examples lead to a fundamental question in the study of operator theoretic dynamics, 
which was already hinted toward in the example of the Duffing equation. When the 
physically relevant (pseudo)eigenvectors lie outside of the space of study, as e.g. Dirac 
distributions on equillibria and periodic orbits, or e.g. adjoint Hardy space functionals, 
how should one know which approximated eigenvectors are "real"? 

This interesting phenomenon that many discretization techniques are excited by eigenmodes of 
Koopman operators over functions spaces which are \emph{strictly larger} than $L^2$. 
This raises a question - which computed eigenmodes are $(1)$ "expected": 
approximations of $L^2$ Koopman eigenmodes, $(2)$ "unexpected": approximations of 
eigenmodes in unintended function spaces, $(3)$: "spurious": caused solely by the 
discretization?

The current state of research holds two potential answers to this question, the first of 
which has already been mentioned in section \ref{sec:validation}: one can make a 
philosophical change and assume that the underlying dynamics is governed by a 
stochastic system. The result of this assumption is that $\scrK$ (resp. $\scrL$) 
"collapses" to a compact operator. In this case one can construct \emph{strongly} 
convergent approximation schemes e.g. Ulams' method \cite{attr} or point-cloud methods 
based on entropically regularized optimal transport \cite{entropic}. However, the 
stochastic approach "destroys fine structures" in the sense that anythinig which is 
not robust to perturbation is changed, e.g. fixed points become absolutely continuous 
invariant measures with (potentially) support on the entire phase space. 

Another approach \cite{rigged} considers only measure preserving dynamical systems 
(thereby restricting to the case that $\scrK$ is an isometry) but tackles continuous and 
point spectrum simultaneously by embedding some space $\scrX$ into a \emph{rigged Hilbert 
space} structure, $\scrX \subset L^2 (\Omega) \subset \scrX^*$. By carefully choosing this 
approximation space, one can guarantee a existence of a full spectral decomposition of 
$\scrK$. 


% -------------------------------------------------------------------------------------- %
