\documentclass[a4paper, draft]{article}
\usepackage{amsmath}
\usepackage{parskip}

\begin{document}


\textbf{Question 1 } Observe the following polyhedron $P^{\leq}(A, b)$ with 

\begin{equation*} 
    A = \begin{pmatrix}
        1 & \cdot & \cdot & \cdot & \cdot & \cdot & \cdot \\
        \cdot & 1 & \cdot & \cdot & \cdot & \cdot & \cdot \\
        \cdot & \cdot & 1 & \cdot & \cdot & \cdot & \cdot \\
        \cdot & \cdot & \cdot & 1 & \cdot & \cdot & \cdot \\
        \cdot & \cdot & \cdot & \cdot & 1 & 1 & 1 \\
        \cdot & \cdot & \cdot & \cdot & 2 & 2 & 2 \\
        \cdot & \cdot & \cdot & \cdot & 3 & 3 & 3 
    \end{pmatrix}
    , \quad b = \begin{pmatrix}
        2 \\
        4 \\
        6 \\
        1 \\
        2 \\
        8 \\
        1 
    \end{pmatrix}.
\end{equation*}
\textit{a)} Determine $lineal(P)$.

\textit{b)} What is the affine dimension of a minimal face in $P$? \newline\newline

\textit{Solution} 
\begin{equation*}
    lineal(P) = ker(A) = \text{span}\{
    \begin{pmatrix}
        \cdot \\
        \cdot \\
        \cdot \\
        \cdot \\
        1 \\
        -1 \\
        \cdot 
    \end{pmatrix}, \begin{pmatrix}
        \cdot \\
        \cdot \\
        \cdot \\
        \cdot \\
        1 \\
        \cdot \\
        -1
    \end{pmatrix}\}
\end{equation*}

For a minimal face $F$ we have $F = x + lineal(P)$ for any $x \in F$. Since $lineal(P)$ has dimension 2, a minimal face will have affine dimension 2. \newpage

\textbf{Question 2 } A Hamilton cycle is a cycle in a graph $G$ that visits every vertex. 

Formulate an integer program for a given graph $G = (V, E)$ to determine if the graph contains a hamilton cycle. 

\textit{Hint: A graph is connected if for any bipartition of the vertices, there must be an edge running between the bipartition.} \newline\newline

\textit{Solution } We wish to find a subset of edges that form a Hamilton cycle, so we define our variable $x \in \{0, 1\}^E$. This is a viability problem, so the objective function is unimportant. A Hamilton cycle must satisfy two constraints:
\begin{itemize}
    \item The induced graph must be a cycle. For a given vertex, exactly one edge must "enter" the vertex, and one must "exit" it. To model this:
    \begin{equation*}
        \sum_{e\, =\, (v, w)} x_e = 2, \quad \forall\, v \in V,
    \end{equation*}
    \item The induced graph must be connected. For any bipartition of the vertices, there must be edges that run between the bipartition. As with H4.2:
    \begin{equation*}
        \sum_{\substack{e\, =\, (v, w) \\ v \in S \\ w \notin S}} x_e \geq 2, \quad \forall\, S \subset V,\ S \notin \{ \emptyset , V \}.
    \end{equation*}
\end{itemize}

This model does cause the same issue as in H4.2, which is that it can require an exponential number of constraints. 


\end{document}